\providecommand\ifstandalone[1]{#1}
\ifstandalone{
\documentclass[9pt]{memoir}
\usepackage{graphicx}
\usepackage[hidelinks]{hyperref}
\usepackage[paperwidth=4in,paperheight=7in,top=.25in,bottom=.25in,inner=.25in,outer=.25in,includeheadfoot]{geometry}
\usepackage{enumitem}
\newcommand*{\blankpage}{%
\vspace*{\fill}
{\centering This page intentionally left blank \\ except for the presence of this message.\par}
\vspace{\fill}}
\makeatletter
\renewcommand*{\cleardoublepage}{\cleartorecto}
\renewcommand*{\cleartorecto}{\clearpage\if@twoside \ifodd\c@page\else
\blankpage
\thispagestyle{empty}
\newpage
\if@twocolumn\hbox{}\newpage\fi\fi\fi}
\renewcommand*{\cleartoverso}{\clearpage\if@twoside \ifodd\c@page
\blankpage
\thispagestyle{empty}
\newpage
\if@twocolumn\hbox{}\newpage\fi\else\fi\fi}
\makeatother

\renewcommand{\thepart}{\arabic{part}}
% \renewcommand{\thechapter} this will be defined per-book
\renewcommand{\thesection}{\thechapter.\ifnum\value{section}<10 0\fi\arabic{section}}
\renewcommand{\thefigure}{\thechapter-\arabic{figure}}
\renewcommand{\theparagraph}{\ifnum\value{paragraph}<10 0\fi\arabic{paragraph}}
\counterwithin*{chapter}{part}
\counterwithin*{part}{book}
\setlength{\cftbooknumwidth}{0pc}
\setlength{\cftbookindent}{0pc}
\renewcommand{\booknumberlinebox}[2]{#2}
\renewcommand*{\cftbookaftersnum}{:\\}
\setlength{\cftpartnumwidth}{0pc}
\setlength{\cftpartindent}{0pc}
\renewcommand{\partnumberlinebox}[2]{#2}
\renewcommand*{\cftpartname}{Part\space}
\renewcommand*{\cftpartaftersnum}{:\space}
\setlength{\cftchapternumwidth}{2pc}
\setlength{\cftsectionnumwidth}{3pc}
\setlength{\cftsectionindent}{0pc}
\renewcommand{\cftdot}{\hspace{.5pc}.\hspace{-.5pc}}
\makeatletter
\renewcommand{\@pnumwidth}{1.5pc}
\renewcommand{\@tocrmarg}{1.5pc}
\makeatother

\setlist[enumerate,1]{noitemsep,label={\Alph*.}}
\setlist[enumerate,2]{noitemsep,label={\arabic*.}}
\setlength{\emergencystretch}{3pt}

\newcommand\setnext[2]{\setcounter{#1}{#2}\addtocounter{#1}{-1}}
\newcommand\letterval[1]{%
\if#1A1%
\else\if#1B2%
\else\if#1C3%
\else\if#1D4%
\else\if#1E5%
\else\if#1F6%
\else\if#1G7%
\else\if#1H8%
\else\if#1I9%
\else\if#1J10%
\else\if#1K11%
\else\if#1L12%
\else\if#1M13%
\else\if#1N14%
\else\if#1O15%
\else\if#1P16%
\else\if#1Q17%
\else\if#1R18%
\else\if#1S19%
\else\if#1T20%
\else\if#1U21%
\else\if#1V22%
\else\if#1W23%
\else\if#1X24%
\else\if#1Y25%
\else\if#1Z26%
\else0%
\fi\fi\fi\fi\fi\fi\fi\fi\fi\fi\fi\fi\fi\fi\fi\fi\fi\fi\fi\fi\fi\fi\fi\fi\fi\fi}

\newif\ifshowstatus
\showstatustrue
\newcommand{\status}[1]{\ifshowstatus\section*{Status}\par#1\par\fi}

\renewcommand{\booknamefont}{\sffamily\huge\bfseries}
\renewcommand{\booknumfont}{\sffamily\huge\bfseries}
\renewcommand{\booktitlefont}{\sffamily\Huge\mdseries}
\renewcommand{\bookname}{}
\renewcommand{\beforebookskip}{\null\vfil\noindent\hrulefill\vfil}
%\renewcommand{\midbookskip}{\par\vskip 2\onelineskip}
\renewcommand{\afterbookskip}{\vfil\noindent\hrulefill\vfil\newpage\blankpage}
\renewcommand{\partnamefont}{\sffamily\huge\mdseries}
\renewcommand{\partnumfont}{\sffamily\huge\mdseries}
\renewcommand{\parttitlefont}{\sffamily\Huge\mdseries}
\renewcommand{\beforepartskip}{\null\vfil\noindent\hrulefill\vfil}
\renewcommand{\afterpartskip}{\vfil\vfil\newpage\blankpage}
\renewcommand{\chapnamefont}{\sffamily\huge\mdseries\itshape}
\renewcommand{\chapnumfont}{\sffamily\huge\mdseries\itshape}
\renewcommand{\chaptitlefont}{\sffamily\Huge\mdseries\upshape}
\setsecheadstyle{\sffamily\Large\mdseries}

\let\oldcontentsline\contentsline
\renewcommand\contentsline[4]{
\def\eqtesta{#1}
\ifx\eqtesta\figureliteral
\ifx\lofbook\thisbook
\oldcontentsline{#1}{#2}{#3}{#4}
\fi
\else
\oldcontentsline{#1}{#2}{#3}{#4}
\fi
}
\newcommand\startlofdomain[1]{
\write1{\string\@writefile{lof}{\string\def\string\lofbook{#1}}}
\def\thisbook{#1}
}
\def\figureliteral{figure}

\newcommand{\cfR}{23 CFR 655.603}
\newcommand{\cfra}{23 CFR 655.603(a)}
\newcommand{\cfrf}{23 Code of Federal Regulations (CFR), Part 655, Subpart F}



\renewcommand{\thechapter}{\thepart\Alph{chapter}}

\setsecnumdepth{subparagraph}
\maxsecnumdepth{subparagraph}

\newcounter{mpartype}[section]
\newcommand{\mpara}{\vspace{.5\baselineskip}\refstepcounter{paragraph}}
\newcommand{\standard}[1]{\mpara\noindent{\ifnum\thempartype=1\relax\else{}}{\bfseries{}Standard:}\\\fi\setcounter{mpartype}{1}\theparagraph\quad{\bfseries#1}}
\newcommand{\option}[1]{\mpara\noindent{}\ifnum\thempartype=2\relax\else{}Option:\\\fi\setcounter{mpartype}{2}\theparagraph\quad{#1}}
\newcommand{\support}[1]{\mpara\noindent{}\ifnum\thempartype=3\relax\else{}Support:\\\fi\setcounter{mpartype}{3}\theparagraph\quad{#1}}
\newcommand{\guidance}[1]{\mpara\noindent{}\ifnum\thempartype=4\relax\else{}{\itshape{}Guidance:}\\\fi\setcounter{mpartype}{4}\theparagraph\quad{\itshape#1}}
\newif\ifshowstatus
\showstatustrue
\newcommand{\status}[1]{\ifshowstatus\section*{Status}\par#1\par\fi}

\begin{document}
\setcounter{page}{9001}
\setnext{part}{1}
\part{General}
\setnext{chapter}{\letterval{A}}
}

\chapter{General}

\status{Check for things that need to be changed for Centralia setting in first two sections; fill in remaining sections.}

\section{Purpose of Traffic Control Devices}

\support{
The purpose of traffic control devices, as well as the principles for their use, is to promote highway safety and efficiency by providing for the orderly movement of all road users on streets, highways, bikeways, and private roads open to public travel throughout the Nation.}

\support{
Traffic control devices notify road users of regulations and provide warning and guidance needed for the uniform and efficient operation of all elements of the traffic stream in a manner intended to minimize the occurrences of crashes.}

\standard{
Traffic control devices or their supports shall not bear any advertising message or any other message that is not related to traffic control.}

\support{
Tourist-oriented directional signs and Specific Service signs are not considered advertising; rather, they are classified as motorist service signs.}

\section{Principles of Traffic Control Devices}

\support{
This Manual contains the basic principles that govern the design and use of traffic control devices for all streets, highways, bikeways, and private roads open to public travel (see definition in Section~\ref{sec:2009.1A.13} on page~\pageref{sec:2009.1A.13}) regardless of type or class or the public agency, official, or owner having jurisdiction. This Manual's text specifies the restriction on the use of a device if it is intended for limited application or for a specific system. It is important that these principles be given primary consideration in the selection and application of each device.}

\guidance{
To be effective, a traffic control device should meet five basic requirements:
\begin{enumerate}
\item Fulfill a need;
\item Command attention;
\item Convey a clear, simple meaning;
\item Command respect from road users; and
\item Give adequate time for proper response.
\end{enumerate}}

\guidance{
Design, placement, operation, maintenance, and uniformity are aspects that should be carefully considered in order to maximize the ability of a traffic control device to meet the five requirements listed in the previous paragraph. Vehicle speed should be carefully considered as an element that governs the design, operation, placement, and location of various traffic control devices.}

\support{
The definition of the word ``speed'' varies depending on its use. The definitions of specific speed terms are contained in Section~\ref{sec:2009.1A.13} on page~\pageref{sec:2009.1A.13}.}

\guidance{
The actions required of road users to obey regulatory devices should be specified by State statute, or in cases not covered by State statute, by local ordinance or resolution. Such statutes, ordinances, and resolutions should be consistent with the ``Uniform Vehicle Code'' (see Section~\ref{sec:2009.1A.11} on page~\pageref{sec:2009.1A.11}).}

\guidance{
The proper use of traffic control devices should provide the reasonable and prudent road user with the information necessary to efficiently and lawfully use the streets, highways, pedestrian facilities, and bikeways.}

\support{
Uniformity of the meaning of traffic control devices is vital to their effectiveness. The meanings ascribed to devices in this Manual are in general accord with the publications mentioned in Section~\ref{sec:2009.1A.11} on page~\pageref{sec:2009.1A.11}.}

\section{Design of Traffic Control Devices}

\guidance{
Devices should be designed so that features such as size, shape, color, composition, lighting or retroreflection, and contrast are combined to draw attention to the devices; that size, shape, color, and simplicity of message combine to produce a clear meaning; that legibility and size combine with placement to permit adequate time for response; and that uniformity, size, legibility, and reasonableness of the message combine to command respect.}

\guidance{
Aspects of a device's standard design should be modified only if there is a demonstrated need.}

\support{
An example of modifying a device's design would be to modify the Combination Horizontal Alignment/Intersection (W1-10) sign to show intersecting side roads on both sides rather than on just one side of the major road within the curve.}

\option{
With the exception of symbols and colors, minor modifications in the specific design elements of a device may be made provided the essential appearance characteristics are preserved.}

\section{Placement and Operation of Traffic Control Devices}

\guidance{
Placement of a traffic control device should be within the road user's view so that adequate visibility is provided. To aid in conveying the proper meaning, the traffic control device should be appropriately positioned with respect to the location, object, or situation to which it applies. The location and legibility of the traffic control device should be such that a road user has adequate time to make the proper response in both day and night conditions.}

\guidance{
Traffic control devices should be placed and operated in a uniform and consistent manner.}

\guidance{
Unnecessary traffic control devices should be removed. The fact that a device is in good physical condition should not be a basis for deferring needed removal or change.}

\section{Maintenance of Traffic Control Devices}

\guidance{
Functional maintenance of traffic control devices should be used to determine if certain devices need to be changed to meet current traffic conditions.}

\guidance{
Physical maintenance of traffic control devices should be performed to retain the legibility and visibility of the device, and to retain the proper functioning of the device.}

\support{
Clean, legible, properly mounted devices in good working condition command the respect of road users.}

\section{Uniformity of Traffic Control Devices}

\support{
Uniformity of devices simplifies the task of the road user because it aids in recognition and understanding, thereby reducing perception/reaction time. Uniformity assists road users, law enforcement officers, and traffic courts by giving everyone the same interpretation. Uniformity assists public highway officials through efficiency in manufacture, installation, maintenance, and administration. Uniformity means treating similar situations in a similar way. The use of uniform traffic control devices does not, in itself, constitute uniformity. A standard device used where it is not appropriate is as objectionable as a non-standard device; in fact, this might be worse, because such misuse might result in disrespect at those locations where the device is needed and appropriate.}

\section{Responsibility for Traffic Control Devices}

\standard{
The responsibility for the design, placement, operation, maintenance, and uniformity of traffic control devices shall rest with the public agency or the official having jurisdiction, or, in the case of private roads open to public travel, with the private owner or private official having jurisdiction. \cfR{} adopts the MUTCD as the national standard for all traffic control devices installed on any street, highway, bikeway, or private road open to public travel  (see definition in Section~\ref{sec:2009.1A.13} on page~\pageref{sec:2009.1A.13}). When a State or other Federal agency manual or supplement is required, that manual or supplement shall be in substantial conformance with the National MUTCD.}

\standard{
\cfR{} also states that traffic control devices on all streets, highways, bikeways, and private roads open to public travel in each State shall be in substantial conformance with standards issued or endorsed by the Federal Highway Administrator.}

\support{
The Introduction of this Manual contains information regarding the meaning of substantial conformance and the applicability of the MUTCD to private roads open to public travel.}

\support{
The ``Uniform Vehicle Code'' (see Section~\ref{sec:2009.1A.11} on page~\pageref{sec:2009.1A.11}) has the following provision in Section 15-104 for the adoption of a uniform manual:

\begin{quote}
    ``The [State Highway Agency] shall adopt a manual and specification for a uniform system of traffic control devices consistent with the provisions of this code for use upon highways within this State. Such uniform system shall correlate with and so far as possible conform to the system set forth in the most recent edition of the Manual on Uniform Traffic Control Devices for Streets and Highways, and other standards issued or endorsed by the Federal Highway Administrator.''
    
    ``The Manual adopted pursuant to subsection (a) shall have the force and effect of law.''
\end{quote}}

\support{
All States have officially adopted the National MUTCD either in its entirety, with supplemental provisions, or as a separate published document.}

\guidance{
These individual State manuals or supplements should be reviewed for specific provisions relating to that State.}

\support{
The National MUTCD has also been adopted by the National Park Service, the U.S. Forest Service, the U.S. Military Command, the Bureau of Indian Affairs, the Bureau of Land Management, and the U.S. Fish and Wildlife Service.}

\guidance{
States should adopt Section 15-116 of the ``Uniform Vehicle Code,'' which states that, ``No person shall install or maintain in any area of private property used by the public any sign, signal, marking, or other device intended to regulate, warn, or guide traffic unless it conforms with the State manual and specifications adopted under Section 15-104.''}

\section{Authority for Placement of Traffic Control Devices}

\standard{
Traffic control devices, advertisements, announcements, and other signs or messages within the highway right-of-way shall be placed only as authorized by a public authority or the official having jurisdiction, or, in the case of private roads open to public travel, by the private owner or private official having jurisdiction, for the purpose of regulating, warning, or guiding traffic.}

\standard{
When the public agency or the official having jurisdiction over a street or highway or, in the case of private roads open to public travel, the private owner or private official having jurisdiction, has granted proper authority, others such as contractors and public utility companies shall be permitted to install temporary traffic control devices in temporary traffic control zones. Such traffic control devices shall conform with the Standards of this Manual.}

\standard{
All regulatory traffic control devices shall be supported by laws, ordinances, or regulations.}

\support{
Provisions of this Manual are based upon the concept that effective traffic control depends upon both appropriate application of the devices and reasonable enforcement of the regulations.}

\support{
Although some highway design features, such as curbs, median barriers, guardrails, speed humps or tables, and textured pavement, have a significant impact on traffic operations and safety, they are not considered to be traffic control devices and provisions regarding their design and use are generally not included in this Manual.}

\support{
Certain types of signs and other devices that do not have any traffic control purpose are sometimes placed within the highway right-of-way by or with the permission of the public agency or the official having jurisdiction over the street or highway. Most of these signs and other devices are not intended for use by road users in general, and their message is only important to individuals who have been instructed in their meanings. These signs and other devices are not considered to be traffic control devices and provisions regarding their design and use are not included in this Manual. Among these signs and other devices are the following:

\begin{enumerate}
   \item Devices whose purpose is to assist highway maintenance personnel. Examples include markers to guide snowplow operators, devices that identify culvert and drop inlet locations, and devices that precisely identify highway locations for maintenance or mowing purposes.
   \item Devices whose purpose is to assist fire or law enforcement personnel. Examples include markers that identify fire hydrant locations, signs that identify fire or water district boundaries, speed measurement pavement markings, small indicator lights to assist in enforcement of red light violations, and photo enforcement systems.
   \item Devices whose purpose is to assist utility company personnel and highway contractors, such as markers that identify underground utility locations.
   \item Signs posting local non-traffic ordinances.
   \item Signs giving civic organization meeting information.
\end{enumerate}}

\standard{
Signs and other devices that do not have any traffic control purpose that are placed within the highway right-of-way shall not be located where they will interfere with, or detract from, traffic control devices.}

\guidance{
Any unauthorized traffic control device or other sign or message placed on the highway right-of-way by a private organization or individual constitutes a public nuisance and should be removed. All unofficial or non-essential traffic control devices, signs, or messages should be removed.}

\section{Engineering Study and Engineering Judgment}

\support{
Definitions of an engineering study and engineering judgment are contained in in Section~\ref{sec:2009.1A.13} on page~\pageref{sec:2009.1A.13}.}

\standard{
This Manual describes the application of traffic control devices, but shall not be a legal requirement for their installation.}

\guidance{
The decision to use a particular device at a particular location should be made on the basis of either an engineering study or the application of engineering judgment. Thus, while this Manual provides Standards, Guidance, and Options for design and applications of traffic control devices, this Manual should not be considered a substitute for engineering judgment. Engineering judgment should be exercised in the selection and application of traffic control devices, as well as in the location and design of roads and streets that the devices complement.}

\guidance{
Early in the processes of location and design of roads and streets, engineers should coordinate such location and design with the design and placement of the traffic control devices to be used with such roads and streets.}

\guidance{
Jurisdictions, or owners of private roads open to public travel, with responsibility for traffic control that do not have engineers on their staffs who are trained and/or experienced in traffic control devices should seek engineering assistance from others, such as the State transportation agency, their county, a nearby large city, or a traffic engineering consultant.}

\support{
As part of the Federal-aid Program, each State is required to have a Local Technology Assistance Program (LTAP) and to provide technical assistance to local highway agencies. Requisite technical training in the application of the principles of the MUTCD is available from the State's Local Technology Assistance Program for needed engineering guidance and assistance.}

\section{Interpretations, Experimentations, Changes, and Interim Approvals}

section

\section{Relation to Other Publications}
\label{sec:2009.1A.11}

section

\section{Color Code}

section

\section{Definitions of Headings, Words, and Phrases in this Manual}
\label{sec:2009.1A.13}
\status{Fill in remainder of this section.}

\standard{
\label{p:2009.1A.13.01}
When used in this Manual, the text headings of Standard, Guidance, Option, and Support shall be defined as follows:
\begin{enumerate}
\item Standard---a statement of required, mandatory, or specifically prohibitive practice regarding a traffic control device. All Standard statements are labeled, and the text appears in bold type. The verb ``shall'' is typically used. The verbs ``should'' and ``may'' are not used in Standard statements. Standard statements are sometimes modified by Options.
\item Guidance---a statement of recommended, but not mandatory, practice in typical situations, with deviations allowed if engineering judgment or engineering study indicates the deviation to be appropriate. All Guidance statements are labeled, and the text appears in unbold type. The verb ``should'' is typically used. The verbs ``shall'' and ``may'' are not used in Guidance statements. Guidance statements are sometimes modified by Options.
\item Option---a statement of practice that is a permissive condition and carries no requirement or recommendation. Option statements sometime contain allowable modifications to a Standard or Guidance statement. All Option statements are labeled, and the text appears in unbold type. The verb ``may'' is typically used. The verbs ``shall'' and ``should'' are not used in Option statements.
\item Support---an informational statement that does not convey any degree of mandate, recommendation, authorization, prohibition, or enforceable condition. Support statements are labeled, and the text appears in unbold type. The verbs ``shall,'' ``should,'' and ``may'' are not used in Support statements.
\end{enumerate}}


\section{Meanings of Acronyms and Abbreviations in this Manual}

section

\section{Abbreviations Used on Traffic Control Devices}

section

\ifstandalone{\end{document}}