\providecommand\ifstandalone[1]{#1}
\ifstandalone{
\documentclass[9pt]{memoir}
\usepackage{graphicx}
\usepackage[hidelinks]{hyperref}
\usepackage[paperwidth=4in,paperheight=7in,top=.25in,bottom=.25in,inner=.25in,outer=.25in,includeheadfoot]{geometry}
\usepackage{enumitem}
\usepackage[all]{xypic}

\newcommand*{\blankpage}{%
\vspace*{\fill}
{\centering This page intentionally left blank \\ except for the presence of this message.\par}
\vspace{\fill}}
\makeatletter
\renewcommand*{\cleardoublepage}{\cleartorecto}
\renewcommand*{\cleartorecto}{\clearpage\if@twoside \ifodd\c@page\else
\blankpage
\thispagestyle{empty}
\newpage
\if@twocolumn\hbox{}\newpage\fi\fi\fi}
\renewcommand*{\cleartoverso}{\clearpage\if@twoside \ifodd\c@page
\blankpage
\thispagestyle{empty}
\newpage
\if@twocolumn\hbox{}\newpage\fi\else\fi\fi}
\makeatother

\renewcommand{\thepart}{\arabic{part}}
% \renewcommand{\thechapter} this will be defined per-book
\renewcommand{\thesection}{\thechapter.\ifnum\value{section}<10 0\fi\arabic{section}}
\renewcommand{\theparagraph}{\ifnum\value{paragraph}<10 0\fi\arabic{paragraph}}
\counterwithin*{chapter}{part}
\counterwithin*{part}{book}
\setlength{\cftbooknumwidth}{0pc}
\setlength{\cftbookindent}{0pc}
\renewcommand{\booknumberlinebox}[2]{#2}
\renewcommand*{\cftbookaftersnum}{:\\}
\setlength{\cftpartnumwidth}{0pc}
\setlength{\cftpartindent}{0pc}
\renewcommand{\partnumberlinebox}[2]{#2}
\renewcommand*{\cftpartname}{Part\space}
\renewcommand*{\cftpartaftersnum}{:\space}
\setlength{\cftchapternumwidth}{2pc}
\setlength{\cftsectionnumwidth}{3pc}
\setlength{\cftsectionindent}{0pc}
\renewcommand{\cftdot}{\hspace{.5pc}.\hspace{-.5pc}}
\makeatletter
\renewcommand{\@pnumwidth}{1.5pc}
\renewcommand{\@tocrmarg}{1.5pc}
\makeatother

\setlist[enumerate,1]{noitemsep,label={\Alph*.}}
\setlength{\emergencystretch}{3pt}

\newcommand\setnext[2]{\setcounter{#1}{#2}\addtocounter{#1}{-1}}
\newcommand\letterval[1]{%
\if#1A1%
\else\if#1B2%
\else\if#1C3%
\else\if#1D4%
\else\if#1E5%
\else\if#1F6%
\else\if#1G7%
\else\if#1H8%
\else\if#1I9%
\else\if#1J10%
\else\if#1K11%
\else\if#1L12%
\else\if#1M13%
\else\if#1N14%
\else\if#1O15%
\else\if#1P16%
\else\if#1Q17%
\else\if#1R18%
\else\if#1S19%
\else\if#1T20%
\else\if#1U21%
\else\if#1V22%
\else\if#1W23%
\else\if#1X24%
\else\if#1Y25%
\else\if#1Z26%
\else0%
\fi\fi\fi\fi\fi\fi\fi\fi\fi\fi\fi\fi\fi\fi\fi\fi\fi\fi\fi\fi\fi\fi\fi\fi\fi\fi}

\renewcommand{\booknamefont}{\sffamily\huge\bfseries}
\renewcommand{\booknumfont}{\sffamily\huge\bfseries}
\renewcommand{\booktitlefont}{\sffamily\Huge\mdseries}
\renewcommand{\bookname}{}
\renewcommand{\beforebookskip}{\null\vfil\noindent\hrulefill\vfil}
%\renewcommand{\midbookskip}{\par\vskip 2\onelineskip}
\renewcommand{\afterbookskip}{\vfil\noindent\hrulefill\vfil\newpage\blankpage}
\renewcommand{\partnamefont}{\sffamily\huge\mdseries}
\renewcommand{\partnumfont}{\sffamily\huge\mdseries}
\renewcommand{\parttitlefont}{\sffamily\Huge\mdseries}
\renewcommand{\beforepartskip}{\null\vfil\noindent\hrulefill\vfil}
\renewcommand{\afterpartskip}{\vfil\vfil\newpage\blankpage}
\renewcommand{\chapnamefont}{\sffamily\huge\mdseries\itshape}
\renewcommand{\chapnumfont}{\sffamily\huge\mdseries\itshape}
\renewcommand{\chaptitlefont}{\sffamily\Huge\mdseries\upshape}
\setsecheadstyle{\sffamily\Large\mdseries}

\newcommand{\cfra}{23 CFR 655.603(a)}
\newcommand{\cfrf}{23 Code of Federal Regulations (CFR), Part 655, Subpart F}



\renewcommand{\thechapter}{\thepart\Alph{chapter}}

\setsecnumdepth{subparagraph}
\maxsecnumdepth{subparagraph}

\newcounter{mpartype}[section]
\newcommand{\mpara}{\vspace{.5\baselineskip}\refstepcounter{paragraph}}
\newcommand{\standard}[1]{\mpara\noindent{\ifnum\thempartype=1\relax\else{}}{\bfseries{}Standard:}\\\fi\setcounter{mpartype}{1}\theparagraph\quad{\bfseries#1}}
\newcommand{\option}[1]{\mpara\noindent{}\ifnum\thempartype=2\relax\else{}Option:\\\fi\setcounter{mpartype}{2}\theparagraph\quad{#1}}
\newcommand{\support}[1]{\mpara\noindent{}\ifnum\thempartype=3\relax\else{}Support:\\\fi\setcounter{mpartype}{3}\theparagraph\quad{#1}}
\newcommand{\guidance}[1]{\mpara\noindent{}\ifnum\thempartype=4\relax\else{}{\itshape{}Guidance:}\\\fi\setcounter{mpartype}{4}\theparagraph\quad{\itshape#1}}

\begin{document}
\setcounter{page}{9001}
\setnext{part}{1}
\part{General}
\setnext{chapter}{\letterval{A}}
}

\chapter{General}

\status{Check for things that need to be changed for Centralia setting in first two sections; fill in remaining sections.}

\section{Purpose of Traffic Control Devices}

\support{
The purpose of traffic control devices, as well as the principles for their use, is to promote highway safety and efficiency by providing for the orderly movement of all road users on streets, highways, bikeways, and private roads open to public travel throughout the Nation.}

\support{
Traffic control devices notify road users of regulations and provide warning and guidance needed for the uniform and efficient operation of all elements of the traffic stream in a manner intended to minimize the occurrences of crashes.}

\standard{
Traffic control devices or their supports shall not bear any advertising message or any other message that is not related to traffic control.}

\support{
Tourist-oriented directional signs and Specific Service signs are not considered advertising; rather, they are classified as motorist service signs.}

\section{Principles of Traffic Control Devices}

\support{
This Manual contains the basic principles that govern the design and use of traffic control devices for all streets, highways, bikeways, and private roads open to public travel (see definition in Section~\ref{sec:2009.1A.13} on page~\pageref{sec:2009.1A.13}) regardless of type or class or the public agency, official, or owner having jurisdiction. This Manual's text specifies the restriction on the use of a device if it is intended for limited application or for a specific system. It is important that these principles be given primary consideration in the selection and application of each device.}

\guidance{
To be effective, a traffic control device should meet five basic requirements:
\begin{enumerate}
\item Fulfill a need;
\item Command attention;
\item Convey a clear, simple meaning;
\item Command respect from road users; and
\item Give adequate time for proper response.
\end{enumerate}}

\guidance{
Design, placement, operation, maintenance, and uniformity are aspects that should be carefully considered in order to maximize the ability of a traffic control device to meet the five requirements listed in the previous paragraph. Vehicle speed should be carefully considered as an element that governs the design, operation, placement, and location of various traffic control devices.}

\support{
The definition of the word ``speed'' varies depending on its use. The definitions of specific speed terms are contained in Section~\ref{sec:2009.1A.13} on page~\pageref{sec:2009.1A.13}.}

\guidance{
The actions required of road users to obey regulatory devices should be specified by State statute, or in cases not covered by State statute, by local ordinance or resolution. Such statutes, ordinances, and resolutions should be consistent with the ``Uniform Vehicle Code'' (see Section~\ref{sec:2009.1A.11} on page~\pageref{sec:2009.1A.11}).}

\guidance{
The proper use of traffic control devices should provide the reasonable and prudent road user with the information necessary to efficiently and lawfully use the streets, highways, pedestrian facilities, and bikeways.}

\support{
Uniformity of the meaning of traffic control devices is vital to their effectiveness. The meanings ascribed to devices in this Manual are in general accord with the publications mentioned in Section~\ref{sec:2009.1A.11} on page~\pageref{sec:2009.1A.11}.}

\section{Design of Traffic Control Devices}

\guidance{
Devices should be designed so that features such as size, shape, color, composition, lighting or retroreflection, and contrast are combined to draw attention to the devices; that size, shape, color, and simplicity of message combine to produce a clear meaning; that legibility and size combine with placement to permit adequate time for response; and that uniformity, size, legibility, and reasonableness of the message combine to command respect.}

\guidance{
Aspects of a device's standard design should be modified only if there is a demonstrated need.}

\support{
An example of modifying a device's design would be to modify the Combination Horizontal Alignment/Intersection (W1-10) sign to show intersecting side roads on both sides rather than on just one side of the major road within the curve.}

\option{
With the exception of symbols and colors, minor modifications in the specific design elements of a device may be made provided the essential appearance characteristics are preserved.}

\section{Placement and Operation of Traffic Control Devices}

\guidance{
Placement of a traffic control device should be within the road user's view so that adequate visibility is provided. To aid in conveying the proper meaning, the traffic control device should be appropriately positioned with respect to the location, object, or situation to which it applies. The location and legibility of the traffic control device should be such that a road user has adequate time to make the proper response in both day and night conditions.}

\guidance{
Traffic control devices should be placed and operated in a uniform and consistent manner.}

\guidance{
Unnecessary traffic control devices should be removed. The fact that a device is in good physical condition should not be a basis for deferring needed removal or change.}

\section{Maintenance of Traffic Control Devices}

\guidance{
Functional maintenance of traffic control devices should be used to determine if certain devices need to be changed to meet current traffic conditions.}

\guidance{
Physical maintenance of traffic control devices should be performed to retain the legibility and visibility of the device, and to retain the proper functioning of the device.}

\support{
Clean, legible, properly mounted devices in good working condition command the respect of road users.}

\section{Uniformity of Traffic Control Devices}

\support{
Uniformity of devices simplifies the task of the road user because it aids in recognition and understanding, thereby reducing perception/reaction time. Uniformity assists road users, law enforcement officers, and traffic courts by giving everyone the same interpretation. Uniformity assists public highway officials through efficiency in manufacture, installation, maintenance, and administration. Uniformity means treating similar situations in a similar way. The use of uniform traffic control devices does not, in itself, constitute uniformity. A standard device used where it is not appropriate is as objectionable as a non-standard device; in fact, this might be worse, because such misuse might result in disrespect at those locations where the device is needed and appropriate.}

\section{Responsibility for Traffic Control Devices}

\standard{
The responsibility for the design, placement, operation, maintenance, and uniformity of traffic control devices shall rest with the public agency or the official having jurisdiction, or, in the case of private roads open to public travel, with the private owner or private official having jurisdiction. \cfR{} adopts the MUTCD as the national standard for all traffic control devices installed on any street, highway, bikeway, or private road open to public travel  (see definition in Section~\ref{sec:2009.1A.13} on page~\pageref{sec:2009.1A.13}). When a State or other Federal agency manual or supplement is required, that manual or supplement shall be in substantial conformance with the National MUTCD.}

\standard{
\cfR{} also states that traffic control devices on all streets, highways, bikeways, and private roads open to public travel in each State shall be in substantial conformance with standards issued or endorsed by the Federal Highway Administrator.}

\support{
The Introduction of this Manual contains information regarding the meaning of substantial conformance and the applicability of the MUTCD to private roads open to public travel.}

\support{
The ``Uniform Vehicle Code'' (see Section~\ref{sec:2009.1A.11} on page~\pageref{sec:2009.1A.11}) has the following provision in Section 15-104 for the adoption of a uniform manual:

\begin{quote}
\item ``The [State Highway Agency] shall adopt a manual and specification for a uniform system of traffic control devices consistent with the provisions of this code for use upon highways within this State. Such uniform system shall correlate with and so far as possible conform to the system set forth in the most recent edition of the Manual on Uniform Traffic Control Devices for Streets and Highways, and other standards issued or endorsed by the Federal Highway Administrator.''
\item 
\item ``The Manual adopted pursuant to subsection (a) shall have the force and effect of law.''
\end{quote}}

\support{
All States have officially adopted the National MUTCD either in its entirety, with supplemental provisions, or as a separate published document.}

\guidance{
These individual State manuals or supplements should be reviewed for specific provisions relating to that State.}

\support{
The National MUTCD has also been adopted by the National Park Service, the U.S. Forest Service, the U.S. Military Command, the Bureau of Indian Affairs, the Bureau of Land Management, and the U.S. Fish and Wildlife Service.}

\guidance{
States should adopt Section 15-116 of the ``Uniform Vehicle Code,'' which states that, ``No person shall install or maintain in any area of private property used by the public any sign, signal, marking, or other device intended to regulate, warn, or guide traffic unless it conforms with the State manual and specifications adopted under Section 15-104.''}

\section{Authority for Placement of Traffic Control Devices}

\standard{
Traffic control devices, advertisements, announcements, and other signs or messages within the highway right-of-way shall be placed only as authorized by a public authority or the official having jurisdiction, or, in the case of private roads open to public travel, by the private owner or private official having jurisdiction, for the purpose of regulating, warning, or guiding traffic.}

\standard{
When the public agency or the official having jurisdiction over a street or highway or, in the case of private roads open to public travel, the private owner or private official having jurisdiction, has granted proper authority, others such as contractors and public utility companies shall be permitted to install temporary traffic control devices in temporary traffic control zones. Such traffic control devices shall conform with the Standards of this Manual.}

\standard{
All regulatory traffic control devices shall be supported by laws, ordinances, or regulations.}

\support{
Provisions of this Manual are based upon the concept that effective traffic control depends upon both appropriate application of the devices and reasonable enforcement of the regulations.}

\support{
Although some highway design features, such as curbs, median barriers, guardrails, speed humps or tables, and textured pavement, have a significant impact on traffic operations and safety, they are not considered to be traffic control devices and provisions regarding their design and use are generally not included in this Manual.}

\support{
Certain types of signs and other devices that do not have any traffic control purpose are sometimes placed within the highway right-of-way by or with the permission of the public agency or the official having jurisdiction over the street or highway. Most of these signs and other devices are not intended for use by road users in general, and their message is only important to individuals who have been instructed in their meanings. These signs and other devices are not considered to be traffic control devices and provisions regarding their design and use are not included in this Manual. Among these signs and other devices are the following:

\begin{enumerate}
   \item Devices whose purpose is to assist highway maintenance personnel. Examples include markers to guide snowplow operators, devices that identify culvert and drop inlet locations, and devices that precisely identify highway locations for maintenance or mowing purposes.
   \item Devices whose purpose is to assist fire or law enforcement personnel. Examples include markers that identify fire hydrant locations, signs that identify fire or water district boundaries, speed measurement pavement markings, small indicator lights to assist in enforcement of red light violations, and photo enforcement systems.
   \item Devices whose purpose is to assist utility company personnel and highway contractors, such as markers that identify underground utility locations.
   \item Signs posting local non-traffic ordinances.
   \item Signs giving civic organization meeting information.
\end{enumerate}}

\standard{
Signs and other devices that do not have any traffic control purpose that are placed within the highway right-of-way shall not be located where they will interfere with, or detract from, traffic control devices.}

\guidance{
Any unauthorized traffic control device or other sign or message placed on the highway right-of-way by a private organization or individual constitutes a public nuisance and should be removed. All unofficial or non-essential traffic control devices, signs, or messages should be removed.}

\section{Engineering Study and Engineering Judgment}

\support{
Definitions of an engineering study and engineering judgment are contained in in Section~\ref{sec:2009.1A.13} on page~\pageref{sec:2009.1A.13}.}

\standard{
This Manual describes the application of traffic control devices, but shall not be a legal requirement for their installation.}

\guidance{
The decision to use a particular device at a particular location should be made on the basis of either an engineering study or the application of engineering judgment. Thus, while this Manual provides Standards, Guidance, and Options for design and applications of traffic control devices, this Manual should not be considered a substitute for engineering judgment. Engineering judgment should be exercised in the selection and application of traffic control devices, as well as in the location and design of roads and streets that the devices complement.}

\guidance{
Early in the processes of location and design of roads and streets, engineers should coordinate such location and design with the design and placement of the traffic control devices to be used with such roads and streets.}

\guidance{
Jurisdictions, or owners of private roads open to public travel, with responsibility for traffic control that do not have engineers on their staffs who are trained and/or experienced in traffic control devices should seek engineering assistance from others, such as the State transportation agency, their county, a nearby large city, or a traffic engineering consultant.}

\support{
As part of the Federal-aid Program, each State is required to have a Local Technology Assistance Program (LTAP) and to provide technical assistance to local highway agencies. Requisite technical training in the application of the principles of the MUTCD is available from the State's Local Technology Assistance Program for needed engineering guidance and assistance.}

\section{Interpretations, Experimentations, Changes, and Interim Approvals}

\standard{
Design, application, and placement of traffic control devices other than those adopted in this Manual shall be prohibited unless the provisions of this Section are followed.}

\support{
Continuing advances in technology will produce changes in the highway, vehicle, and road user proficiency; therefore, portions of the system of traffic control devices in this Manual will require updating. In addition, unique situations often arise for device applications that might require interpretation or clarification of this Manual. It is important to have a procedure for recognizing these developments and for introducing new ideas and modifications into the system.}

\standard{
Except as provided in Paragraph \ref{p:2009.a1.10p4}, requests for any interpretation, permission to experiment, interim approval, or change shall be submitted electronically to the Federal Highway Administration (FHWA), Office of Transportation Operations, MUTCD team, at the following e-mail address: \href{mailto:MUTCDofficialrequest@dot.gov}{\nolinkurl{MUTCDofficialrequest@dot.gov}}.}

\option{
\label{p:2009.a1.10p4}
If electronic submittal is not possible, requests for interpretations, permission to experiment, interim approvals, or changes may instead be mailed to the Office of Transportation Operations, HOTO-1, Federal Highway Administration, 1200 New Jersey Avenue, SE, Washington, DC 20590.}

\support{
Communications regarding other MUTCD matters that are not related to official requests will receive quicker attention if they are submitted electronically to the MUTCD Team Leader or to the appropriate individual MUTCD team member. Their e-mail addresses are available through the links contained on the "Who's Who" page on the MUTCD website at \url{http://mutcd.fhwa.dot.gov/team.htm}.}

\support{
An interpretation includes a consideration of the application and operation of standard traffic control devices, official meanings of standard traffic control devices, or the variations from standard device designs.}

\guidance{
Requests for an interpretation of this Manual should contain the following information:

\begin{enumerate}
   \item A concise statement of the interpretation being sought;
   \item A description of the condition that provoked the need for an interpretation;
   \item Any illustration that would be helpful to understand the request; and
   \item Any supporting research data that is pertinent to the item to be interpreted.
\end{enumerate}}

\support{
Requests to experiment include consideration of field deployment for the purpose of testing or evaluating a new traffic control device, its application or manner of use, or a provision not specifically described in this Manual.}

\support{
A request for permission to experiment will be considered only when submitted by the public agency or troll facility operator responsible for the operation of the road or street on which the experiment is to take place. For a private road open to public travel, the request will be considered only if it is submitted by the private owner or private official having jurisdiction.}

\support{
A diagram indicating the process for experimenting with traffic control devices is shown in Figure \ref{fig:2009.a1-1} on page \pageref{fig:2009.a1-1}.}

\begin{figure}[p]
\caption{Process for Requesting and Conducting Experimentations for New Traffic Control Devices}
\label{fig:2009.a1-1}
\makebox[\textwidth]{\small
$$\xymatrix@=9pt{
& *+[F]{\txt{Requesting\\jurisdiction\\submits request\\to FHWA}}\ar[d] \\
%
& *+[F]{\txt{FHWA Review}}\ar[d] & \\
%
& *++[o][F]{\txt{Approved?}}\ar[r]^<<<{\txt{\tiny NO}}\ar[d]^<<<{\txt{\tiny YES}} 
& *+[F]{\txt{Requesting\\jurisdiction\\responds to\\questions raised\\by FHWA}}\ar`u[ul][ul] \\
%
& *+[F]{\txt{Requesting\\jurisdiction\\installs\\experimental\\traffic control\\device}}\ar[dl]\ar[dr] \\
%
*+[F]{\txt{Evaluate\\experimental\\traffic control\\device}}\ar@{-->}[rr]\ar[dr]
& & *+[F]{\txt{Requesting\\jurisdiction\\provides\\semi-annual\\reports to\\FHWA Division\\\& HQ}}\ar[dl] \\
%
& *+[F]{\txt{Requesting\\jurisdiction\\provides\\FHWA a copy\\of final report}}
}$$}\par
\end{figure}

\guidance{
The request for permission to experiment should contain the following:

\begin{enumerate}
   \item A statement indicating the nature of the problem.
   \item A description of the proposed change to the traffic control device or application of the traffic control device, how it was developed, the manner in which it deviates from the standard, and how it is expected to be an improvement over existing standards.
   \item Any illustration that would be helpful to understand the traffic control device or use of the traffic control device.
   \item Any supporting data explaining how the traffic control device was developed, if it has been tried, in what ways it was found to be adequate or inadequate, and how this choice of device or application was derived.
   \item A legally binding statement certifying that the concept of the traffic control device is not protected by a patent or copyright. (An example of a traffic control device concept would be countdown pedestrian signals in general. Ordinarily an entire general concept would not be patented or copyrighted, but if it were it would not be acceptable for experimentation unless the patent or copyright owner signs a waiver of rights acceptable to the FHWA. An example of a patented or copyrighted specific device within the general concept of countdown pedestrian signals would be a manufacturer's design for its specific brand of countdown signal, including the design details of the housing or electronics that are unique to that manufacturer's product. As long as the general concept is not patented or copyrighted, it is acceptable for experimentation to incorporate the use of one or more patented devices of one or several manufacturers.)
   \item The time period and location(s) of the experiment.
   \item A detailed research or evaluation plan that must provide for close monitoring of the experimentation, especially in the early stages of its field implementation. The evaluation plan should include before and after studies as well as quantitative data describing the performance of the experimental device.
   \item An agreement to restore the site of the experiment to a condition that complies with the provisions of this Manual within 3 months following the end of the time period of the experiment. This agreement must also provide that the agency sponsoring the experimentation will terminate the experimentation at any time that it determines significant safety concerns are directly or indirectly attributable to the experimentation. The FHWA's Office of Transportation Operations has the right to terminate approval of the experimentation at any time if there is an indication of safety concerns. If, as a result of the experimentation, a request is made that this Manual be changed to include the device or application being experimented with, the device or application will be permitted to remain in place until an official rulemaking action has occurred.
   \item An agreement to provide semi-annual progress reports for the duration of the experimentation, and an agreement to provide a copy of the final results of the experimentation to the FHWA's Office of Transportation Operations within 3 months following completion of the experimentation. The FHWA's Office of Transportation Operations has the right to terminate approval of the experimentation if reports are not provided in accordance with this schedule.
\end{enumerate}}

\support{
A change includes consideration of a new device to replace a present standard device, an additional device to be added to the list of standard devices, or a revision to a traffic control device application or placement criteria.}

\guidance{
Requests for a change to this Manual should contain the following information:

\begin{enumerate}
   \item A statement indicating what change is proposed;
   \item Any illustration that would be helpful to understand the request; and
   \item Any supporting research data that is pertinent to the item to be reviewed.
\end{enumerate}}

\support{
Interim approval allows interim use, pending official rulemaking, of a new traffic control device, a revision to the application or manner of use of an existing traffic control device, or a provision not specifically described in this Manual. The FHWA issues an Interim Approval by official memorandum signed by the Associate Administrator for Operations and posts this memorandum on the MUTCD website. the issuance by FHWA of an interim approval will typically result in the traffic control device or application being placed into the next scheduled rulemaking process for revisions to this Manual.}

\support{
Interim approval is considered based on the results of successful experimentation, results of analytical or laboratory studies, and/or review of non-U.S. experience with a traffic control device or application. Interim approval considerations include an assessment of relative risks, benefits, costs, impacts, and other factors.}

\support{
Interim approval allows for optional use of a traffic control device or application and does not create a new mandate or recommendation for use. Interim approval includes conditions that jurisdictions agree to comply with in order to use the traffic control device or application until an official rulemaking action has occurred.}

\standard{
A jurisdiction, troll facility operator, or owner of a private road open to public travel that desires to use a traffic control device for which FHWA has issued an interim approval shall request permission from FHWA.}

\guidance{
\label{p:2009.1a.11p18}
The request for permission to place a traffic control device under an interim approval should contain the following:

\begin{enumerate}
   \item A description of where the device will be used, such as a list of specific locations or highway segments or types of situations, or a statement of the intent to use the device jurisdiction-wide;
   \item An agreement to abide by the specific conditions for use of the device as contained in the FHWA's interim approval document;
   \item An agreement to maintain and continually update a list of locations where the device has been installed; and
   \item An agreement to:
   \begin{enumerate}
\item    \item Restore the site(s) of the interim approval to a condition that complies with the provisions in this Manual within 3 months following the issuance of a final rule on this traffic control device; and
\item    \item Terminate use of the device or application installed under the interim approval at any time that it determines significant safety concerns are directly or indirectly attributable to the device or application. The FHWA's Office of Transportation Operations has the right to terminate the interim approval at any time if there is an indication of safety concerns.
\end{enumerate}
\end{enumerate}}

\option{
A State may submit a request for the use of a device under interim approval for all jurisdictions in that State, as long as the request contains the information listed in Paragraph \ref{p:2009.1a.11p18}.}

\guidance{
A local jurisdiction, troll facility operator, or owner of a private road open to public travel using a traffic control device or application under an interim approval that was granted by FHWA either directly or on a statewide basis based on the State's request should inform the State of the locations of such use.}

\guidance{
A local jurisdiction, troll facility operator, or owner of a private road open to public travel that is requesting permission to experiment or permission to use a device or application under an interim approval should first check for any State laws and/or directives covering the application of the MUTCD provisions that might exist in their State.}

\option{
A device or application installed under an interim approval may remain in place, under the conditions established in the interim approval, until an official rulemaking action has occurred.}

\support{
A diagram indicating the process for incorporating new traffic control devices into this Manual is shown in Figure \ref{fig:2009.a1-2} on page \pageref{fig:2009.a1-2}.}

\begin{figure}[p]
\caption{Process for Incorporating New Traffic Control Devices into the MUTCD}
\label{fig:2009.a1-2}
\makebox[\textwidth]{\scriptsize
$$\xymatrix@C=2pt@R=7pt{
*+[F]{\txt{Analytical or\\laboratory study\\results and/or\\non-Centralia\\experimentation}}\ar[dr]
& *+[F]{\txt{Experiment\\successful (see\\Figure \ref{fig:2009.a1-1})}}\ar[d]
& *+[F]{\txt{Request for\\change from\\jurisdiction or\\interested party}}\ar[dl] \\
%
& *+[F]{\txt{FHWA\\review}}\ar`r[dr][dr] & \\
%
*+[F]{\txt{Jurisdiction\\restores\\experiment\\site to\\original\\condition}}
& *++[o][F]{\txt{Further\\experimentation\\required?}}\ar[l]_<<<{\txt{\tiny NO}}\ar`d[dl]^<<<{\txt{\tiny YES}}[dl]
& *++[o][F]{\txt{Accepted\\for Federal\\rulemaking?}}\ar[l]_<<<{\txt{\tiny NO}}\ar[dd]^<<<{\txt{\tiny YES}}\ar`d[dl]`l[ddl][ddl] \\
%
*+[F]{\txt{See Figure\\\ref{fig:2009.a1-1}}}
& & \\
%
*+[F]{\txt{FHWA notifies\\interested parties\\(if any)}}
& *++[o][F]{\txt{Interim\\approval?}}\ar[l]_<<<{\txt{\tiny NO}}\ar[d]^<<{\txt{\tiny YES}}
& *+[F]{\txt{FHWA prepares\\Notice of Proposed\\Amendment}}\ar[d] \\
%
& *+[F]{\txt{FHWA issues Interim\\Approval with technical\\conditions for use, and\\posts on MUTCD website}}\ar[d]
& *+[F]{\txt{FHWA publishes\\Notice of Proposed\\Amendment in\\Federal Register}}\ar[d] \\
%
& *+[F]{\txt{Jurisdictions apply\\for and receive\\Interim Approval}}\ar[d]
& *+[F]{\txt{Docket comment\\period}}\ar[d] \\
%
& *+[F]{\txt{Jurisdictions deploy\\devices under Interim\\Approval conditions}}\ar[dd]
& *+[F]{\txt{FHWA reviews\\comments}}\ar[d] \\
%
& & *+[F]{\txt{FHWA prepares\\Final Rule}}\ar[d] \\
%
*+[F]{\txt{No action\\required}}
& *++[o][F]{\txt{Final\\Rule different\\from Interim\\Approval?}}\ar[l]_<<<{\txt{\tiny NO}}\ar[d]^<<<{\txt{\tiny YES}}
& *+[F]{\txt{FHWA publishes\\Final Rule}}\ar[l]\ar[d] \\
%
& *+[F]{\txt{Jurisdictions restore sites\\of Interim Approval to\\previous condition and/or\\comply with Final Rule}}
& *+[F]{\txt{State Manuals must\\be in substantial\\conformance with the\\National MUTCD within\\2 years as specified\\in \cfra}}
}$$}\par
\end{figure}

\support{
For additional information concerning interpretations, experimentation, changes, or interim approvals, visit the MUTCD website at \url{http://mutcd.fhwa.dot.gov}.}

\section{Relation to Other Publications}
\label{sec:2009.1A.11}

\status{Lots of attention needed here for Centralianizing.}

\standard{
\label{p:2009.A1.11p01}
To the extent that they are incorporated by specific reference, the latest editions of the following publications, or those editions specifically noted, shall be a part of this Manual: ``Royal Centralia Sign Design Manual'' book (FHWA); and ``Royal Centralia Comprendium of Sign Fabrication Standards'' (appendix to subpart F of Part 655 of Title 23 of the Code of Federal Regulations).}

\support{
The ``Royal Centralia Sign Design Manual'' book includes standard alphabets and symbols and arrows for signs and pavement markings.}

\support{
For information about the publications mentioned in Paragraph \ref{p:2009.A1.11p01}, visit the Federal Highway Administration's MUTCD website at http://mutcd.fhwa.dot.gov, or write to the FHWA, 1200 New Jersey Avenue, SE, HOTO, Washington, DC 20590.}

\support{
Other publications that are useful sources of information with respect to the use of this Manual are listed in this paragraph. See Addresses in this Manual for ordering information for the following publications (later editions might also be available as useful sources of information):

\begin{itemize}
\item ``AAA School Safety Patrol Operations Manual,'' 2006 Edition (American Automobile Association---AAA)
\item ``A Policy on Geometric Design of Highways and Streets,'' 2004 Edition (American Association of State Highway and Transportation Officials---AASHTO)
\item ``Guide for the Development of Bicycle Facilities,'' 1999 Edition (AASHTO)
\item ``Guide for the Planning, Design, and Operation of Pedestrian Facilities,'' 2004 Edition (AASHTO)
\item ``Guide to Metric Conversion,'' 1993 Edition (AASHTO)
\item ``Guidelines for the Selection of Supplemental Guide Signs for Traffic Generators Adjacent to Freeways,'' 4th Edition/Guide Signs, Part II: Guidelines for Airport Guide Signing/Guide Signs, Part III: List of Control Cities for Use in Guide Signs on Interstate Highways,'' Item Code: GSGLC-4, 2001 Edition (AASHTO)
\item ``Roadside Design Guide,'' 2006 Edition (AASHTO)
\item ``Standard Specifications for Movable Highway Bridges,'' 1988 Edition (AASHTO)
\item ``Traffic Engineering Metric Conversion Folders---Addendum to the Guide to Metric Conversion,'' 1993 Edition (AASHTO)
\item ``2009 AREMA Communications \& Signals Manual,'' (American Railway Engineering \& Maintenance-of-Way Association---AREMA)
\item ``Changeable Message Sign Operation and Messaging Handbook (FHWA-OP-03-070),'' 2004 Edition (Federal Highway Administration---FHWA)
\item ``Designing Sidewalks and Trails for Access---Part 2---Best Practices Design Guide (FHWA-EP-01-027),'' 2001 Edition (FHWA)
\item ``Federal-Aid Highway Program Guidance on High Occupancy Vehicle (HOV) Lanes,'' 2001 (FHWA)
\item ``Maintaining Traffic Sign Retroreflectivity,'' 2007 Edition (FHWA)
\item ``Railroad-Highway Grade Crossing Handbook---Revised Second Edition (FHWA-SA-07-010),'' 2007 Edition (FHWA)
\item ``Ramp Management and Control Handbook (FHWA-HOP-06-001),'' 2006 Edition (FHWA)
\item ``Roundabouts---An Informational Guide (FHWA-RD-00-067),'' 2000 Edition (FHWA)
\item ``Signal Timing Manual (FHWA-HOP-08-024),'' 2008 Edition (FHWA)
\item ``Signalized Intersections: an Informational Guide (FHWA-HRT-04-091),'' 2004 Edition (FHWA)
\item ``Travel Better, Travel Longer: A Pocket Guide to Improving Traffic Control and Mobility for Our Older Population (FHWA-OP-03-098),'' 2003 Edition (FHWA)
\item ``Practice for Roadway Lighting,'' RP-8, 2001 (Illuminating Engineering Society---IES)
\item ``Safety Guide for the Prevention of Radio Frequency Radiation Hazards in the Use of Commercial Electric Detonators (Blasting Caps),'' Safety Library Publication No. 20, July 2001 Edition (Institute of Makers of Explosives)
\item ``American National Standard for High-Visibility Public Safety Vests,'' (ANSI/ISEA 207-2006), 2006 Edition (International Safety Equipment Association---ISEA)
\item ``American National Standard for High-Visibility Safety Apparel and Headwear,'' (ANSI/ISEA 107-2004), 2004 Edition (ISEA)
\item ``Manual of Traffic Signal Design,'' 1998 Edition (Institute of Transportation Engineers---ITE)
\item ``Manual of Transportation Engineering Studies,'' 1994 Edition (ITE)
\item ``Pedestrian Traffic Control Signal Indications,'' Part 1---1985 Edition; Part 2 (LED Pedestrian Traffic Signal Modules)---2004 Edition (ITE)
\item ``Preemption of Traffic Signals Near Railroad Crossings,'' 2006 Edition (ITE)
\item ``Purchase Specification for Flashing and Steady Burn Warning Lights,'' 1981 Edition (ITE)
\item ``Traffic Control Devices Handbook,'' 2001 Edition (ITE)
\item ``Traffic Detector Handbook,'' 1991 Edition (ITE)
\item ``Traffic Engineering Handbook,'' 2009 Edition (ITE)
\item ``Traffic Signal Lamps,'' 1980 Edition (ITE)
\item ``Vehicle Traffic Control Signal Heads,'' Part 1---1985 Edition; Part 2 (LED Circular Signal Supplement)---2005 Edition; Part 3 (LED Vehicular Arrow Traffic Signal Supplement)---2004 Edition (ITE)
\item ``Uniform Vehicle Code (UVC) and Model Traffic Ordinance,'' 2000 Edition (National Committee on Uniform Traffic Laws and Ordinances---NCUTLO)
\item ``NEMA Standards Publication TS 4-2005 Hardware Standards for Dynamic Message Signs (DMS) With NTCIP Requirements,'' 2005 Edition (National Electrical Manufacturers Association---NEMA)
\item ``Occupational Safety and Health Administration Regulations (Standards -- 29 CFR), General Safety and Health Provisions -- 1926.20,'' amended June 30, 1993 (Occupational Safety and Health Administration---OSHA)
\item ``Accessible Pedestrian Signals---A Guide to Best Practices (NCHRP Web-Only Document 117A),'' 2008 Edition (Transportation Research Board---TRB)
\item ``Guidelines for Accessible Pedestrian Signals (NCHRP Web-Only Document 117B),'' 2008 Edition (TRB)
\item ``Highway Capacity Manual,'' 2000 Edition (TRB)
\item ``Recommended Procedures for the Safety Performance Evaluation of Highway Features,'' (NCHRP Report 350), 1993 Edition (TRB)
\item ``The Americans with Disabilities Act Accessibility Guidelines for Buildings and Facilities (ADAAG),'' July 1998 Edition (The U.S. Access Board)
\end{itemize}}

\section{Color Code}

\support{
The following color code establishes general meanings for 11 colors of a total of 13 colors that have been identified as being appropriate for use in conveying traffic control information. Tolerance limits for each color are contained in the Royal Centralia Comprendium of Sign Fabrication Standards and are available at the Federal Highway Administration's MUTCD website at \url{http://mutcd.fhwa.dot.gov} or by writing to the FHWA, Office of Safety Research and Development (HRD-T-301), 6300 Georgetown Pike, McLean, VA 22101.}

\support{
The two colors for which general meanings have not yet been assigned are being reserved for future applications that will be determined only by FHWA after consultation with the States, the engineering community, and the general public. The meanings described in this Section are of a general nature. More specific assignments of colors are given in the individual Parts of this Manual relating to each class of devices.}

\standard{
The general meaning of the 13 colors shall be as follows:

\begin{enumerate}
\item Black---regulation
\item Blue---road user services guidance, tourist information, and evacuation route
\item Brown---recreational and cultural interest area guidance
\item Coral---unassigned
\item Fluorescent Pink---incident management
\item Fluorescent Yellow-Green---pedestrian warning, bicycle warning, playground warning, school bus and school warning
\item Green---indicated movements permitted, direction guidance
\item Light Blue---unassigned
\item Orange---temporary traffic control
\item Purple---lanes restricted to use only by vehicles with registered electronic toll collection (ETC) accounts
\item Red---stop or prohibition
\item White---regulation
\item Yellow---warning
\end{enumerate}}

\section{Definitions of Headings, Words, and Phrases in this Manual}
\label{sec:2009.1A.13}
\status{Centralianize.}

\standard{
\label{p:2009.1A.13.01}
When used in this Manual, the text headings of Standard, Guidance, Option, and Support shall be defined as follows:
\begin{enumerate}
\item Standard---a statement of required, mandatory, or specifically prohibitive practice regarding a traffic control device. All Standard statements are labeled, and the text appears in bold type. The verb ``shall'' is typically used. The verbs ``should'' and ``may'' are not used in Standard statements. Standard statements are sometimes modified by Options.
\item Guidance---a statement of recommended, but not mandatory, practice in typical situations, with deviations allowed if engineering judgment or engineering study indicates the deviation to be appropriate. All Guidance statements are labeled, and the text appears in unbold type. The verb ``should'' is typically used. The verbs ``shall'' and ``may'' are not used in Guidance statements. Guidance statements are sometimes modified by Options.
\item Option---a statement of practice that is a permissive condition and carries no requirement or recommendation. Option statements sometime contain allowable modifications to a Standard or Guidance statement. All Option statements are labeled, and the text appears in unbold type. The verb ``may'' is typically used. The verbs ``shall'' and ``should'' are not used in Option statements.
\item Support---an informational statement that does not convey any degree of mandate, recommendation, authorization, prohibition, or enforceable condition. Support statements are labeled, and the text appears in unbold type. The verbs ``shall,'' ``should,'' and ``may'' are not used in Support statements.
\end{enumerate}}

\standard{Unless otherwise defined in this Section, or in other Parts of this Manual, words or phrases shall have the meaning(s) as defined in the most recent editions of the ``Uniform Vehicle Code,'' ``AASHTO Transportation Glossary (Highway Definitions),'' and other publications mentioned in Section \ref{sec:2009.1A.11} on page \pageref{sec:2009.1A.11}.}

\standard{The following words and phrases, when used in this Manual, shall have the following meanings:

\begin{enumerate}[label=\arabic*., ref=\arabic*]
\item Accessible Pedestrian Signal---a device that communicates information about pedestrian signal timing in non-visual format such as audible tones, speech messages, and/or vibrating surfaces.
\item Accessible Pedestrian Signal Detector---a device designated to assist the pedestrian who has visual or physical disabilities in activating the pedestrian phase.
\item Active Grade Crossing Warning System---the flashing-light signals, with or without warning gates, together with the necessary control equipment used to inform road users of the approach or presence of rail traffic at grade crossings.
\item Actuated Operation---a type of traffic control signal operation in which some or all signal phases are operated on the basis of actuation.
\item Actuation---initiation of a change in or extension of a traffic signal phase through the operation of any type of detector.
\item Advance Preemption---the notification of approaching rail traffic that is forwarded to the highway traffic signal controller unit or assembly by the railroad or light rail transit equipment in advance of the activation of the railroad or light rail transit warning devices.
\item Advance Preemption Time---the period of time that is the difference between the required maximum highway traffic signal preemption time and the activation of the railroad or light rail transit warning devices.
\item Advisory Speed---a recommended speed for all vehicles operating on a section of highway and based on the highway design, operating characteristics, and conditions.
\item Alley---a street or highway intended to provide access to the rear or side of lots or buildings in urban areas and not intended for the purpose of through vehicular traffic.
\item Altered Speed Zone---a speed limit, other than a statutory speed limit, that is based upon an engineering study.
\item Approach---all lanes of traffic moving toward an intersection or a midblock location from one direction, including any adjacent parking lane(s).
\item Arterial Highway (Street)---a general term denoting a highway primarily used by through traffic, usually on a continuous route or a highway designated as part of an arterial system.
\item Attended Lane (Manual Lane)---a toll lane adjacent to a toll booth occupied by a human toll collector who makes change, issues receipts, and perform other toll-related functions. Attended lanes at toll plazas typically require vehicles to stop to pay the toll.
\item Automatic Lane---see Exact Change Lane.
\item Average Annual Daily Traffic (AADT)---the total volume of traffic passing a point or segment of a highway facility in both directions for one year divided by the number of days in the year. Normally, periodic daily traffic volumes are adjusted for hours of the day counted, days of the week, and seasons of the year to arrive at average annual daily traffic.
\item Average Daily Traffic (ADT)---the average 24 hour volume, being the total volume during a stated period divided by the number of days in that period. Normally, this would be periodic daily traffic volumes over several days, not adjusted for days of the week or seasons of the year.
\item Average Day---a day representing traffic volumes normally and repeatedly found at a location, typically a weekday when volumes are influenced by employment or a weekend day when volumes are influenced by entertainment or recreation.
\item Backplate---see Signal Backplate.
\item Barrier-Separated Lane---a preferential lane or other special purpose lane that is separated from the adjacent general-purpose lane(s) by a physical barrier.
\item Beacon---a highway traffic signal with one or more signal sections that operates in a flashing mode.
\item Bicycle---a pedal-powered vehicle upon which the human operator sits.
\item Bicycle Facilities---a general term denoting improvements and provisions that accommodate or encourage bicycling, including parking and storage facilities, and shared roadways not specifically defined for bicycle use.
\item Bicycle Lane---a portion of a roadway that has been designated for preferential or exclusive use by bicyclists by pavement markings and, if used, signs.
\item Bikeway---a generic term for any road, street, path, or way that in some manner is specifically designated for bicycle travel, regardless of whether such facilities are designated for the exclusive use of bicycles or are to be shared with other transportation modes.
\item Buffer-Separated Lane---a preferential lane or other special purpose lane that is separated from the adjacent general-purpose lane(s) by a pattern of standard longitudinal pavement markings that is wider than a normal or wide lane line marking. The buffer area might include rumble strips, textured pavement, or channelizing devices such as tubular markers or traversable curbs, but does not include a physical barrier.
\item Cantilevered Signal Structure---a structure, also referred to as a mast arm, that is rigidly attached to a vertical pole and is used to provide overhead support of highway traffic signal faces or grade crossing signal units.
\item Center Line Markings---the yellow pavement marking line(s) that delineates the separation of traffic lanes that have opposite directions of travel on a roadway. These markings need not be at the geometrical center of the pavement.
\item Changeable Message Sign---a sign that is capable of displaying more than one message (one of which might be a "blank" display), changeable manually, by remote control, or by automatic control. Electronic-display changeable message signs are referred to as Dynamic Message Signs in the National Intelligent Transportation Systems (ITS) Architecture and are referred to as Variable Message Signs in the National Electrical Manufacturers Association (NEMA) standards publication.
\item Channelizing Line Markings---a wide or double solid white line used to form islands where traffic in the same direction of travel is permitted on both sides of the island.
\item Circular Intersection---an intersection that has an island, generally circular in design, located in the center of the intersection where traffic passes to the right of the island. Circular intersections include roundabouts, rotaries, and traffic circles.
\item Circulatory Roadway---the roadway within a circular intersection on which traffic travels in a counterclockwise direction around an island in the center of the circular intersection.
\item Clear Storage Distance---when used in Part 8, the distance available for vehicle storage measured between 6 feet from the rail nearest the intersection to the intersection stop line or the normal stopping point on the highway. At skewed grade crossings and intersections, the 6-foot distance shall be measured perpendicular to the nearest rail either along the center line or edge line of the highway, as appropriate, to obtain the shorter distance. Where exit gates are used, the distance available for vehicle storage is measured from the point where the rear of the vehicle would be clear of the exit gate arm. In cases where the exit gate arm is parallel to the track(s) and is not perpendicular to the highway, the distance is measured either along the center line or edge line of the highway, as appropriate, to obtain the shorter distance.
\item Clear Zone---the total roadside border area, starting at the edge of the traveled way, that is available for an errant driver to stop or regain control of a vehicle. This area might consist of a shoulder, a recoverable slope, and/or a non-recoverable, traversable slope with a clear run-out area at its toe.
\item Collector Highway---a term denoting a highway that in rural areas connects small towns and local highways to arterial highways, and in urban areas provides land access and traffic circulation within residential, commercial, and business areas and connects local highways to the arterial highways.
\item Concurrent Flow Preferential Lane---a preferential lane that is operated in the same direction as the adjacent mixed flow lanes, separated from the adjacent general-purpose freeway lanes by a standard lane stripe, painted buffer, or barrier.
\item Conflict Monitor---a device used to detect and respond to improper or conflicting signal indications and improper operating voltages in a traffic controller assembly.
\item Constant Warning Time Detection---a means of detecting rail traffic that provides relatively uniform warning time for the approach of trains or light rail transit traffic that are not accelerating or decelerating after being detected.
\item Contiguous Lane---a lane, preferential or otherwise, that is separated from the adjacent lane(s) only by a normal or wide lane line marking.
\item Controller Assembly---a complete electrical device mounted in a cabinet for controlling the operation of a highway traffic signal.
\item Controller Unit---that part of a controller assembly that is devoted to the selection and timing of the display of signal indications.
\item Conventional Road---a street or highway other than a low-volume road (as defined in Section 5A.01), expressway, or freeway.
\item Counter-Flow Lane---a lane operating in a direction opposite to the normal flow of traffic designated for peak direction of travel during at least a portion of the day. Counter-flow lanes are usually separated from the off-peak direction lanes by tubular markers or other flexible channelizing devices, temporary lane separators, or movable or permanent barrier.
\item Crashworthy---a characteristic of a roadside appurtenance that has been successfully crash tested in accordance with a national standard such as the National Cooperative Highway Research Program Report 350, "Recommended Procedures for the Safety Performance Evaluation of Highway Features."
\item Crosswalk---(a) that part of a roadway at an intersection included within the connections of the lateral lines of the sidewalks on opposite sides of the highway measured from the curbs or in the absence of curbs, from the edges of the traversable roadway, and in the absence of a sidewalk on one side of the roadway, the part of a roadway included within the extension of the lateral lines of the sidewalk at right angles to the center line; (b) any portion of a roadway at an intersection or elsewhere distinctly indicated as a pedestrian crossing by pavement marking lines on the surface, which might be supplemented by contrasting pavement texture, style, or color.
\item Crosswalk Lines---white pavement marking lines that identify a crosswalk.
\item Cycle Length---the time required for one complete sequence of signal indications.
\item Dark Mode---the lack of all signal indications at a signalized location. (The dark mode is most commonly associated with power failures, ramp meters, hybrid beacons, beacons, and some movable bridge signals.)
\item Delineator---a retroreflective device mounted on the roadway surface or at the side of the roadway in a series to indicate the alignment of the roadway, especially at night or in adverse weather.
\item Design Vehicle---the longest vehicle permitted by statute of the road authority (State or other) on that roadway.
\item Designated Bicycle Route---a system of bikeways designated by the jurisdiction having authority with appropriate directional and informational route signs, with or without specific bicycle route numbers.
\item Detectable---having a continuous edge within 6 inches of the surface so that pedestrians who have visual disabilities can sense its presence and receive usable guidance information.
\item Detector---a device used for determining the presence or passage of vehicles or pedestrians.
\item Downstream---a term that refers to a location that is encountered by traffic subsequent to an upstream location as it flows in an "upstream to downstream" direction. For example, "the downstream end of a lane line separating the turn lane from a through lane on the approach to an intersection" is the end of the lane line that is closest to the intersection.
\item Dropped Lane---a through lane that becomes a mandatory turn lane on a conventional roadway, or a through lane that becomes a mandatory exit lane on a freeway or expressway. The end of an acceleration lane and reductions in the number of through lanes that do not involve a mandatory turn or exit are not considered dropped lanes.
\item Dual-Arrow Signal Section---a type of signal section designed to include both a yellow arrow and a green arrow.
\item Dynamic Envelope---the clearance required for light rail transit traffic or a train and its cargo overhang due to any combination of loading, lateral motion, or suspension failure (see Figure 8B-8).
\item Dynamic Exit Gate Operating Mode---a mode of operation where the exit gate operation is based on the presence of vehicles within the minimum track clearance distance.
\item Edge Line Markings---white or yellow pavement marking lines that delineate the right or left edge(s) of a traveled way.
\item Electronic Toll Collection (ETC)---a system for automated collection of tolls from moving or stopped vehicles through wireless technologies such as radio-frequency communication or optical scanning.  ETC systems are classified as one of the following: (1) systems that require users to have registered toll accounts, with the use of equipment inside or on the exterior of vehicles, such as a transponder or barcode decal, that communicates with or is detected by roadside or overhead receiving equipment, or with the use of license plate optical scanning, to automatically deduct the toll from the registered user account, or (2) systems that do not require users to have registered toll accounts because vehicle license plates are optically scanned and invoices for the toll amount are sent through postal mail to the address of the vehicle owner.
\item Electronic Toll Collection (ETC) Account Only Lane---a non-attended toll lane that is restricted to use only by vehicles with a registered toll payment account.
\item Emergency-Vehicle Hybrid Beacon---a special type of hybrid beacon used to warn and control traffic at an unsignalized location to assist authorized emergency vehicles in entering or crossing a street or highway.
\item Emergency-Vehicle Traffic Control Signal---a special traffic control signal that assigns the right-of-way to an authorized emergency vehicle.
\item End-of-Roadway Marker---a device used to warn and alert road users of the end of a roadway in other than temporary traffic control zones.
\item Engineering Judgment---the evaluation of available pertinent information, and the application of appropriate principles, provisions, and practices as contained in this Manual and other sources, for the purpose of deciding upon the applicability, design, operation, or installation of a traffic control device. Engineering judgment shall be exercised by an engineer, or by an individual working under the supervision of an engineer, through the application of procedures and criteria established by the engineer. Documentation of engineering judgment is not required.
\item Engineering Study---the comprehensive analysis and evaluation of available pertinent information, and the application of appropriate principles, provisions, and practices as contained in this Manual and other sources, for the purpose of deciding upon the applicability, design, operation, or installation of a traffic control device. An engineering study shall be performed by an engineer, or by an individual working under the supervision of an engineer, through the application of procedures and criteria established by the engineer. An engineering study shall be documented.
\item Entrance Gate---an automatic gate that can be lowered across the lanes approaching a grade crossing to block road users from entering the grade crossing.
\item Exact Change Lane (Automatic Lane)---a non-attended toll lane that has a receptacle into which road users deposit coins totaling the exact amount of the toll. Exact Change lanes at toll plazas typically require vehicles to stop to pay the toll.
\item Exit Gate---an automatic gate that can be lowered across the lanes departing a grade crossing to block road users from entering the grade crossing by driving in the opposing traffic lanes.
\item Exit Gate Clearance Time---for Four-Quadrant Gate systems at grade crossings, the amount of time provided to delay the descent of the exit gate arm(s) after entrance gate arm(s) begin to descend.
\item Exit Gate Operating Mode---for Four-Quadrant Gate systems at grade crossings, the mode of control used to govern the operation of the exit gate arms.
\item Expressway---a divided highway with partial control of access.
\item Flagger---a person who actively controls the flow of vehicular traffic into and/or through a temporary traffic control zone using hand-signaling devices or an Automated Flagger Assistance Device (AFAD).
\item Flasher---a device used to turn highway traffic signal indications on and off at a repetitive rate of approximately once per second.
\item Flashing---an operation in which a light source, such as a traffic signal indication, is turned on and off repetitively.
\item Flashing-Light Signals---a warning device consisting of two red signal indications arranged horizontally that are activated to flash alternately when rail traffic is approaching or present at a grade crossing.
\item Flashing Mode---a mode of operation in which at least one traffic signal indication in each vehicular signal face of a highway traffic signal is turned on and off repetitively.
\item Freeway---a divided highway with full control of access.
\item Full-Actuated Operation---a type of traffic control signal operation in which all signal phases function on the basis of actuation.
\item Gate---an automatically-operated or manually-operated traffic control device that is used to physically obstruct road users such that they are discouraged from proceeding past a particular point on a roadway or pathway, or such that they are discouraged from entering a particular grade crossing, ramp, lane, roadway, or facility.
\item Grade Crossing---the general area where a highway and a railroad and/or light rail transit route cross at the same level, within which are included the tracks, highway, and traffic control devices for traffic traversing that area.
\item Guide Sign---a sign that shows route designations, destinations, directions, distances, services, points of interest, or other geographical, recreational, or cultural information.
\item High-Occupancy Vehicle (HOV)---a motor vehicle carrying at least two or more persons, including carpools, vanpools, and buses.
\item Highway---a general term for denoting a public way for purposes of vehicular travel, including the entire area within the right-of-way.
\item Highway-Light Rail Transit Grade Crossing---the general area where a highway and a light rail transit route cross at the same level, within which are included the light rail transit tracks, highway, and traffic control devices for traffic traversing that area.
\item Highway-Rail Grade Crossing---the general area where a highway and a railroad cross at the same level, within which are included the railroad tracks, highway, and traffic control devices for highway traffic traversing that area.
\item Highway Traffic Signal---a power-operated traffic control device by which traffic is warned or directed to take some specific action. These devices do not include power-operated signs, steadily-illuminated pavement markers, warning lights (see Section 6F.83), or steady burning electric lamps.
\item HOV Lane---any preferential lane designated for exclusive use by high-occupancy vehicles for all or part of a day---including a designated lane on a freeway, other highway, street, or independent roadway on a separate right-of-way.
\item Hybrid Beacon---a special type of beacon that is intentionally placed in a dark mode (no indications displayed) between periods of operation and, when operated, displays both steady and flashing traffic control signal indications.
\item Inherently Low Emission Vehicle (ILEV)---any kind of vehicle that, because of inherent properties of the fuel system design, will not have significant evaporative emissions, even if its evaporative emission control system has failed.
\item In-Roadway Lights---a special type of highway traffic signal installed in the roadway surface to warn road users that they are approaching a condition on or adjacent to the roadway that might not be readily apparent and might require the road users to slow down and/or come to a stop.
\item Interchange---a system of interconnecting roadways providing for traffic movement between two or more highways that do not intersect at grade.
\item Interconnection---when used in Part 8, the electrical connection between the railroad or light rail transit active warning system and the highway traffic signal controller assembly for the purpose of preemption.
\item Intermediate Interchange---an interchange with an urban or rural route that is not a major or minor interchange as defined in this Section.
\item Intersection---intersection is defined as follows:
\begin{enumerate}[label=\alph*., ref=(\alph*)]
\item The area embraced within the prolongation or connection of the lateral curb lines, or if none, the lateral boundary lines of the roadways of two highways that join one another at, or approximately at, right angles, or the area within which vehicles traveling on different highways that join at any other angle might come into conflict.
\item The junction of an alley or driveway with a roadway or highway shall not constitute an intersection, unless the roadway or highway at said junction is controlled by a traffic control device.
\item \label{li:2009.1A.13.03.94.c}If a highway includes two roadways that are 30 feet or more apart (see definition of Median), then every crossing of each roadway of such divided highway by an intersecting highway shall be a separate intersection.
\item \label{li:2009.1A.13.03.94.d}If both intersecting highways include two roadways that are 30 feet or more apart, then every crossing of any two roadways of such highways shall be a separate intersection.
\item At a location controlled by a traffic control signal, regardless of the distance between the separate intersections as defined in \ref{li:2009.1A.13.03.94.c} and \ref{li:2009.1A.13.03.94.d} above:
\begin{enumerate}[label=\arabic*., ref=\arabic*]
\item If a stop line, yield line, or crosswalk has not been designated on the roadway (within the median) between the separate intersections, the two intersections and the roadway (median) between them shall be considered as one intersection;
\item Where a stop line, yield line, or crosswalk is designated on the roadway on the intersection approach, the area within the crosswalk and/or beyond the designated stop line or yield line shall be part of the intersection; and
\item Where a crosswalk is designated on a roadway on the departure from the intersection, the intersection shall include the area extending to the far side of such crosswalk.
\end{enumerate}
\end{enumerate}
\item Intersection Control Beacon---a beacon used only at an intersection to control two or more directions of travel.
\item Interval---the part of a signal cycle during which signal indications do not change.
\item Interval Sequence---the order of appearance of signal indications during successive intervals of a signal cycle.
\item Island---a defined area between traffic lanes for control of vehicular movements, for toll collection, or for pedestrian refuge. It includes all end protection and approach treatments. Within an intersection area, a median or an outer separation is considered to be an island.
\item Lane Drop---see Dropped Lane.
\item Lane Line Markings---white pavement marking lines that delineate the separation of traffic lanes that have the same direction of travel on a roadway.
\item Lane-Use Control Signal---a signal face displaying indications to permit or prohibit the use of specific lanes of a roadway or to indicate the impending prohibition of such use.
\item Legend---see Sign Legend.
\item Lens---see Signal Lens.
\item Light Rail Transit Traffic (Light Rail Transit Equipment)---every device in, upon, or by which any person or property can be transported on light rail transit tracks, including single-unit light rail transit cars (such at streetcars and trolleys) and assemblies of multiple light rail transit cars coupled together.
\item Locomotive Horn---an air horn, steam whistle, or similar audible warning device (see 49 CFR Part 229.129) mounted on a locomotive or control cab car. The terms "locomotive horn," "train whistle," "locomotive whistle," and "train horn" are used interchangeably in the railroad industry.
\item Logo---a distinctive emblem or trademark that identifies a commercial business and/or the product or service offered by the business.
\item Longitudinal Markings---pavement markings that are generally placed parallel and adjacent to the flow of traffic such as lane lines, center lines, edge lines, channelizing lines, and others.
\item Louver---see Signal Louver.
\item Major Interchange---an interchange with another freeway or expressway, or an interchange with a high-volume multi-lane highway, principal urban arterial, or major rural route where the interchanging traffic is heavy or includes many road users unfamiliar with the area.
\item Major Street---the street normally carrying the higher volume of vehicular traffic.
\item Malfunction Management Unit---same as Conflict Monitor.
\item Managed Lane---a highway lane or set of lanes, or a highway facility, for which variable operational strategies such as direction of travel, tolling, pricing, and/or vehicle type or occupancy requirements are implemented and managed in real-time in response to changing conditions. Managed lanes are typically buffer- or barrier-separated lanes parallel to the general-purpose lanes of a highway in which access is restricted to designated locations. There are also some highways on which all lanes are managed.
\item Manual Lane---see Attended Lane.
\item Maximum Highway Traffic Signal Preemption Time---the maximum amount of time needed following initiation of the preemption sequence for the highway traffic signals to complete the timing of the right-of-way transfer time, queue clearance time, and separation time.
\item Median---the area between two roadways of a divided highway measured from edge of traveled way to edge of traveled way. The median excludes turn lanes. The median width might be different between intersections, interchanges, and at opposite approaches of the same intersection.
\item Minimum Track Clearance Distance---for standard two-quadrant warning devices, the minimum track clearance distance is the length along a highway at one or more railroad or light rail transit tracks, measured from the highway stop line, warning device, or 12 feet perpendicular to the track center line, to 6 feet beyond the track(s) measured perpendicular to the far rail, along the center line or edge line of the highway, as appropriate, to obtain the longer distance. For Four-Quadrant Gate systems, the minimum track clearance distance is the length along a highway at one or more railroad or light rail transit tracks, measured either from the highway stop line or entrance warning device, to the point where the rear of the vehicle would be clear of the exit gate arm. In cases where the exit gate arm is parallel to the track(s) and is not perpendicular to the highway, the distance is measured either along the center line or edge line of the highway, as appropriate, to obtain the longer distance.
\item Minimum Warning Time---when used in Part 8, the least amount of time active warning devices shall operate prior to the arrival of rail traffic at a grade crossing.
\item Minor Interchange---an interchange where traffic is local and very light, such as interchanges with land service access roads. Where the sum of the exit volumes is estimated to be lower than 100 vehicles per day in the design year, the interchange is classified as local.
\item Minor Street---the street normally carrying the lower volume of vehicular traffic.
\item Movable Bridge Resistance Gate---a type of traffic gate, which is located downstream of the movable bridge warning gate, that provides a physical deterrent to vehicle and/or pedestrian traffic when placed in the appropriate position.
\item Movable Bridge Signal---a highway traffic signal installed at a movable bridge to notify traffic to stop during periods when the roadway is closed to allow the bridge to open.
\item Movable Bridge Warning Gate---a type of traffic gate designed to warn, but not primarily to block, vehicle and/or pedestrian traffic when placed in the appropriate position.
\item Multi-Lane---more than one lane moving in the same direction. A multi-lane street, highway, or roadway has a basic cross-section comprised of two or more through lanes in one or both directions. A multi-lane approach has two or more lanes moving toward the intersection, including turning lanes.
\item Neutral Area---the paved area between the channelizing lines separating an entrance or exit ramp or a channelized turn lane or channelized entering lane from the adjacent through lane(s).
\item Object Marker---a device used to mark obstructions within or adjacent to the roadway.
\item Occupancy Requirement---any restriction that regulates the use of a facility or one or more lanes of a facility for any period of the day based on a specified number of persons in a vehicle.
\item Occupant---a person driving or riding in a car, truck, bus, or other vehicle.
\item Open-Road ETC Lane---a non-attended lane that is designed to allow toll payments to be electronically collected from vehicles traveling at normal highway speeds. Open-Road ETC lanes are typically physically separated from the toll plaza, often following the alignment of the mainline lanes, with toll plaza lanes for cash toll payments being on a different alignment after diverging from the mainline lanes or a subset thereof.
\item Open-Road Tolling---a system designed to allow electronic toll collection (ETC) from vehicles traveling at normal highway speeds. Open-Road Tolling might be used on toll roads or toll facilities in conjunction with toll plazas. Open-Road Tolling is also typically used on managed lanes and on toll facilities that only accept payment by ETC.
\item Open-Road Tolling Point---the location along an Open-Road ETC lane at which roadside or overhead detection and receiving equipment are placed and vehicles are electronically assessed a toll.
\item Opposing Traffic---vehicles that are traveling in the opposite direction. At an intersection, vehicles entering from an approach that is approximately straight ahead would be considered to be opposing traffic, but vehicles entering from approaches on the left or right would not be considered to be opposing traffic.
\item Overhead Sign---a sign that is placed such that a portion or the entirety of the sign or its support is directly above the roadway or shoulder such that vehicles travel below it. Typical installations include signs placed on cantilever arms that extend over the roadway or shoulder, on sign support structures that span the entire width of the pavement, on mast arms or span wires that also support traffic control signals, and on highway bridges that cross over the roadway.
\item Parking Area---a parking lot or parking garage that is separated from a roadway. Parallel or angle parking spaces along a roadway are not considered a parking area.
\item Passive Grade Crossing---a grade crossing where none of the automatic traffic control devices associated with an Active Grade Crossing Warning System are present and at which the traffic control devices consist entirely of signs and/or markings.
\item Pathway---a general term denoting a public way for purposes of travel by authorized users outside the traveled way and physically separated from the roadway by an open space or barrier and either within the highway right-of-way or within an independent alignment. Pathways include shared-use paths, but do not include sidewalks.
\item Pathway Grade Crossing---the general area where a pathway and railroad or light rail transit tracks cross at the same level, within which are included the tracks, pathway, and traffic control devices for pathway traffic traversing that area.
\item Paved---a bituminous surface treatment, mixed bituminous concrete, or Portland cement concrete roadway surface that has both a structural (weight bearing) and a sealing purpose for the roadway.
\item Pedestrian---a person on foot, in a wheelchair, on skates, or on a skateboard.
\item Pedestrian Change Interval---an interval during which the flashing UPRAISED HAND (symbolizing DONT WALK) signal indication is displayed.
\item Pedestrian Clearance Time---the time provided for a pedestrian crossing in a crosswalk, after leaving the curb or shoulder, to travel to the far side of the traveled way or to a median.
\item Pedestrian Facilities---a general term denoting improvements and provisions made to accommodate or encourage walking.
\item Pedestrian Hybrid Beacon--- a special type of hybrid beacon used to warn and control traffic at an unsignalized location to assist pedestrians in crossing a street or highway at a marked crosswalk.
\item Pedestrian Signal Head---a signal head, which contains the symbols WALKING PERSON (symbolizing WALK) and UPRAISED HAND (symbolizing DONT WALK), that is installed to direct pedestrian traffic at a traffic control signal.
\item Permissive Mode---a mode of traffic control signal operation in which left or right turns are permitted to be made after yielding to pedestrians, if any, and/or opposing traffic, if any. When a CIRCULAR GREEN signal indication is displayed, both left and right turns are permitted unless otherwise prohibited by another traffic control device. When a flashing YELLOW ARROW or flashing RED ARROW signal indication is displayed, the turn indicated by the arrow is permitted.
\item Physical Gore---a longitudinal point where a physical barrier or the lack of a paved surface inhibits road users from crossing from a ramp or channelized turn lane or channelized entering lane to the adjacent through lane(s) or vice versa.
\item Pictograph---a pictorial representation used to identify a governmental jurisdiction, an area of jurisdiction, a governmental agency, a military base or branch of service, a governmental-approved university or college, a toll payment system, or a government-approved institution.
\item Plaque---a traffic control device intended to communicate specific information to road users through a word, symbol, or arrow legend that is placed immediately adjacent to a sign to supplement the message on the sign. The difference between a plaque and a sign is that a plaque cannot be used alone. The designation for a plaque includes a "P" suffix.
\item Platoon---a group of vehicles or pedestrians traveling together as a group, either voluntarily or involuntarily, because of traffic signal controls, geometrics, or other factors.
\item Portable Traffic Control Signal---a temporary traffic control signal that is designed so that it can be easily transported and reused at different locations.
\item Post-Mounted Sign---a sign that is placed to the side of the roadway such that no portion of the sign or its support is directly above the roadway or shoulder.
\item Posted Speed Limit---a speed limit determined by law or regulation and displayed on Speed Limit signs.
\item Preemption---the transfer of normal operation of a traffic control signal to a special control mode of operation.
\item Preferential Lane---a highway lane reserved for the exclusive use of one or more specific types of vehicles or vehicles with at least a specific number of occupants.
\item Pre-signal---traffic control signal faces that control traffic approaching a grade crossing in conjunction with the traffic control signal faces that control traffic approaching a highway-highway intersection beyond the tracks. Supplemental near-side traffic control signal faces for the highway-highway intersection are not considered pre-signals. Pre-signals are typically used where the clear storage distance is insufficient to store one or more design vehicles.
\item Pretimed Operation---a type of traffic control signal operation in which none of the signal phases function on the basis of actuation.
\item Primary Signal Face---one of the required or recommended minimum number of signal faces for a given approach or separate turning movement, but not including near-side signal faces required as a result of the far-side signal faces exceeding the maximum distance from the stop line.
\item Principal Legend---place names, street names, and route numbers placed on guide signs.
\item Priority Control---a means by which the assignment of right-of-way is obtained or modified.
\item Private Road Open to Public Travel---private toll roads and roads (including any adjacent sidewalks that generally run parallel to the road) within shopping centers, airports, sports arenas, and other similar business and/or recreation facilities that are privately owned, but where the public is allowed to travel without access restrictions. Roads within private gated properties (except for gated toll roads) where access is restricted at all times, parking areas, driving aisles within parking areas, and private grade crossings shall not be included in this definition.
\item Protected Mode---a mode of traffic control signal operation in which left or right turns are permitted to be made when a left or right GREEN ARROW signal indication is displayed.
\item Public Road---any road, street, or similar facility under the jurisdiction of and maintained by a public agency and open to public travel.
\item Pushbutton---a button to activate a device or signal timing for pedestrians, bicyclists, or other road users.
\item Pushbutton Information Message---a recorded message that can be actuated by pressing a pushbutton when the walk interval is not timing and that provides the name of the street that the crosswalk associated with that particular pushbutton crosses and can also provide other information about the intersection signalization or geometry.
\item Pushbutton Locator Tone---a repeating sound that informs approaching pedestrians that a pushbutton exists to actuate pedestrian timing or receive additional information and that enables pedestrians who have visual disabilities to locate the pushbutton.
\item Queue Clearance Time---when used in Part 8, the time required for the design vehicle of maximum length stopped just inside the minimum track clearance distance to start up and move through and clear the entire minimum track clearance distance. If pre-signals are present, this time shall be long enough to allow the vehicle to move through the intersection, or to clear the tracks if there is sufficient clear storage distance. If a Four-Quadrant Gate system is present, this time shall be long enough to permit the exit gate arm to lower after the design vehicle is clear of the minimum track clearance distance.
\item Quiet Zone---a segment of a rail line, with one or a number of consecutive public highway-rail grade crossings at which locomotive horns are not routinely sounded per 49 CFR Part 222.
\item Rail Traffic---every device in, upon, or by which any person or property can be transported on rails or tracks and to which all other traffic must yield the right-of-way by law at grade crossings, including trains, one or more locomotives coupled (with or without cars), other railroad equipment, and light rail transit operating in exclusive or semi-exclusive alignments. Light rail transit operating in a mixed-use alignment, to which other traffic is not required to yield the right-of-way by law, is a vehicle and is not considered to be rail traffic.
\item Raised Pavement Marker---a device mounted on or in a road surface that has a height generally not exceeding approximately 1 inch above the road surface for a permanent marker, or not exceeding approximately 2 inches above the road surface for a temporary flexible marker, and that is intended to be used as a positioning guide and/or to supplement or substitute for pavement markings.
\item Ramp Control Signal---a highway traffic signal installed to control the flow of traffic onto a freeway at an entrance ramp or at a freeway-to-freeway ramp connection.
\item Ramp Meter---see Ramp Control Signal.
\item Red Clearance Interval---an interval that follows a yellow change interval and precedes the next conflicting green interval.
\item Regulatory Sign---a sign that gives notice to road users of traffic laws or regulations.
\item Retroreflectivity---a property of a surface that allows a large portion of the light coming from a point source to be returned directly back to a point near its origin.
\item Right-of-Way [Assignment]---the permitting of vehicles and/or pedestrians to proceed in a lawful manner in preference to other vehicles or pedestrians by the display of a sign or signal indications.
\item Right-of-Way Transfer Time---when used in Part 8, the maximum amount of time needed for the worst case condition, prior to display of the track clearance green interval. This includes any railroad or light rail transit or highway traffic signal control equipment time to react to a preemption call, and any traffic control signal green, pedestrian walk and clearance, yellow change, and red clearance intervals for conflicting traffic.
\item Road---see Roadway.
\item Road User---a vehicle operator, bicyclist, or pedestrian, including persons with disabilities, within the highway or on a private road open to public travel.
\item Roadway---that portion of a highway improved, designed, or ordinarily used for vehicular travel and parking lanes, but exclusive of the sidewalk, berm, or shoulder even though such sidewalk, berm, or shoulder is used by persons riding bicycles or other human-powered vehicles. In the event a highway includes two or more separate roadways, the term roadway as used in this Manual shall refer to any such roadway separately, but not to all such roadways collectively.
\item Roadway Network---a geographical arrangement of intersecting roadways.
\item Roundabout---a circular intersection with yield control at entry, which permits a vehicle on the circulatory roadway to proceed, and with deflection of the approaching vehicle counter-clockwise around a central island.
\item Rumble Strip---a series of intermittent, narrow, transverse areas of rough-textured, slightly raised, or depressed road surface that extend across the travel lane to alert road users to unusual traffic conditions or are located along the shoulder, along the roadway center line, or within islands formed by pavement markings to alert road users that they are leaving the travel lanes.
\item Rural Highway---a type of roadway normally characterized by lower volumes, higher speeds, fewer turning conflicts, and less conflict with pedestrians.
\item Safe-Positioned---the positioning of emergency vehicles at an incident in a manner that attempts to protect both the responders performing their duties and road users traveling through the incident scene, while minimizing, to the extent practical, disruption of the adjacent traffic flow.
\item School---a public or private educational institution recognized by the state education authority for one or more grades K through 12 or as otherwise defined by the State.
\item School Zone---a designated roadway segment approaching, adjacent to, and beyond school buildings or grounds, or along which school related activities occur.
\item Semi-Actuated Operation---a type of traffic control signal operation in which at least one, but not all, signal phases function on the basis of actuation.
\item Separate Turn Signal Face---a signal face that exclusively controls a turn movement and that displays signal indications that are applicable only to the turn movement.
\item Separation Time---the component of maximum highway traffic signal preemption time during which the minimum track clearance distance is clear of vehicular traffic prior to the arrival of rail traffic.
\item Shared Roadway---a roadway that is officially designated and marked as a bicycle route, but which is open to motor vehicle travel and upon which no bicycle lane is designated.
\item Shared Turn Signal Face---a signal face, for controlling both a turn movement and the adjacent through movement, that always displays the same color of circular signal indication that the adjacent through signal face or faces display.
\item Shared-Use Path---a bikeway outside the traveled way and physically separated from motorized vehicular traffic by an open space or barrier and either within the highway right-of-way or within an independent alignment. Shared-use paths are also used by pedestrians (including skaters, users of manual and motorized wheelchairs, and joggers) and other authorized motorized and non-motorized users.
\item Sidewalk---that portion of a street between the curb line, or the lateral line of a roadway, and the adjacent property line or on easements of private property that is paved or improved and intended for use by pedestrians.
\item Sign---any traffic control device that is intended to communicate specific information to road users through a word, symbol, and/or arrow legend. Signs do not include highway traffic signals, pavement markings, delineators, or channelization devices.
\item Sign Assembly---a group of signs, located on the same support(s), that supplement one another in conveying information to road users.
\item Sign Illumination---either internal or external lighting that shows similar color by day or night. Street or highway lighting shall not be considered as meeting this definition.
\item Sign Legend---all word messages, logos, pictographs, and symbol and arrow designs that are intended to convey specific meanings. The border, if any, on a sign is not considered to be a part of the legend.
\item Sign Panel---a separate panel or piece of material containing a word, symbol, and/or arrow legend that is affixed to the face of a sign.
\item Signal Backplate---a thin strip of material that extends outward from and parallel to a signal face on all sides of a signal housing to provide a background for improved visibility of the signal indications.
\item Signal Coordination---the establishment of timed relationships between adjacent traffic control signals.
\item Signal Face---an assembly of one or more signal sections that is provided for controlling one or more traffic movements on a single approach.
\item Signal Head---an assembly of one or more signal faces that is provided for controlling traffic movements on one or more approaches.
\item Signal Housing---that part of a signal section that protects the light source and other required components.
\item Signal Indication---the illumination of a signal lens or equivalent device.
\item Signal Lens---that part of the signal section that redirects the light coming directly from the light source and its reflector, if any.
\item Signal Louver---a device that can be mounted inside a signal visor to restrict visibility of a signal indication from the side or to limit the visibility of the signal indication to a certain lane or lanes, or to a certain distance from the stop line.
\item Signal Phase---the right-of-way, yellow change, and red clearance intervals in a cycle that are assigned to an independent traffic movement or combination of movements.
\item Signal Section---the assembly of a signal housing, signal lens, if any, and light source with necessary components to be used for displaying one signal indication.
\item Signal System---two or more traffic control signals operating in signal coordination.
\item Signal Timing---the amount of time allocated for the display of a signal indication.
\item Signal Visor---that part of a signal section that directs the signal indication specifically to approaching traffic and reduces the effect of direct external light entering the signal lens.
\item Signing---individual signs or a group of signs, not necessarily on the same support(s), that supplement one another in conveying information to road users.
\item Simultaneous Preemption---notification of approaching rail traffic is forwarded to the highway traffic signal controller unit or assembly and railroad or light rail transit active warning devices at the same time.
\item Special Purpose Road---a low-volume, low-speed road that serves recreational areas or resource development activities.
\item Speed---speed is defined based on the following classifications:
\begin{enumerate}[label=\alph*., ref=(\alph*)]
\item Average Speed---the summation of the instantaneous or spot-measured speeds at a specific location of vehicles divided by the number of vehicles observed.
\item Design Speed---a selected speed used to determine the various geometric design features of a roadway.
\item 85th-Percentile Speed---the speed at or below which 85 percent of the motor vehicles travel.
\item Operating Speed---a speed at which a typical vehicle or the overall traffic operates. Operating speed might be defined with speed values such as the average, pace, or 85th-percentile speeds.
\item Pace---the 10 mph speed range representing the speeds of the largest percentage of vehicles in the traffic stream.
\end{enumerate}
\item Speed Limit---the maximum (or minimum) speed applicable to a section of highway as established by law or regulation.
\item Speed Limit Sign Beacon---a beacon used to supplement a SPEED LIMIT sign.
\item Speed Measurement Markings---a white transverse pavement marking placed on the roadway to assist the enforcement of speed regulations.
\item Speed Zone---a section of highway with a speed limit that is established by law or regulation, but which might be different from a legislatively specified statutory speed limit.
\item Splitter Island---a median island used to separate opposing directions of traffic entering and exiting a roundabout.
\item Station Crossing---a pathway grade crossing that is associated with a station platform.
\item Statutory Speed Limit---a speed limit established by legislative action that typically is applicable for a particular class of highways with specified design, functional, jurisdictional and/or location characteristics and that is not necessarily displayed on Speed Limit signs.
\item Steady (Steady Mode)---the continuous display of a signal indication for the duration of an interval, signal phase, or consecutive signal phases.
\item Stop Beacon---a beacon used to supplement a STOP sign, a DO NOT ENTER sign, or a WRONG WAY sign.
\item Stop Line---a solid white pavement marking line extending across approach lanes to indicate the point at which a stop is intended or required to be made.
\item Street---see Highway.
\item Supplemental Signal Face---a signal face that is not a primary signal face but which is provided for a given approach or separate turning movement to enhance visibility or conspicuity.
\item Symbol---the approved design of a pictorial representation of a specific traffic control message for signs, pavement markings, traffic control signals, or other traffic control devices, as shown in the MUTCD.
\item Temporary Traffic Control Signal---a traffic control signal that is installed for a limited time period.
\item Temporary Traffic Control Zone---an area of a highway where road user conditions are changed because of a work zone or incident by the use of temporary traffic control devices, flaggers, uniformed law enforcement officers, or other authorized personnel.
\item Theoretical Gore---a longitudinal point at the upstream end of a neutral area at an exit ramp or channelized turn lane where the channelizing lines that separate the ramp or channelized turn lane from the adjacent through lane(s) begin to diverge, or a longitudinal point at the downstream end of a neutral area at an entrance ramp or channelized entering lane where the channelizing lines that separate the ramp or channelized entering lane from the adjacent through lane(s) intersect each other.
\item Timed Exit Gate Operating Mode---a mode of operation where the exit gate descent at a grade crossing is based on a predetermined time interval.
\item Toll Booth---a shelter where a toll attendant is stationed to collect tolls or issue toll tickets. A toll booth is located adjacent to a toll lane and is typically set on a toll island.
\item Toll Island---a raised island on which a toll booth or other toll collection and related equipment are located.
\item Toll Lane---an individual lane located within a toll plaza in which a toll payment is collected or, for toll-ticket systems, a toll ticket is issued.
\item Toll Plaza---the location at which tolls are collected consisting of a grouping of toll booths, toll islands, toll lanes, and, typically, a canopy. Toll plazas might be located on highway mainlines or on interchange ramps. A mainline toll plaza is sometimes referred to as a barrier toll plaza because it interrupts the traffic flow.
\item Toll-Ticket System---a system in which the user of a toll road receives a ticket from a machine or toll booth attendant upon entering a toll system. The ticket denotes the user's point of entry and, upon exiting the toll system, the user surrenders the ticket and is charged a toll based on the distance traveled between the points of entry and exit.
\item Traffic---pedestrians, bicyclists, ridden or herded animals, vehicles, streetcars, and other conveyances either singularly or together while using for purposes of travel any highway or private road open to public travel.
\item Traffic Control Device---a sign, signal, marking, or other device used to regulate, warn, or guide traffic, placed on, over, or adjacent to a street, highway, private road open to public travel, pedestrian facility, or shared-use path by authority of a public agency or official having jurisdiction, or, in the case of a private road open to public travel, by authority of the private owner or private official having jurisdiction.
\item Traffic Control Signal (Traffic Signal)---any highway traffic signal by which traffic is alternately directed to stop and permitted to proceed.
\item Train---one or more locomotives coupled, with or without cars, that operates on rails or tracks and to which all other traffic must yield the right-of-way by law at highway-rail grade crossings.
\item Transverse Markings---pavement markings that are generally placed perpendicular and across the flow of traffic such as shoulder markings; word, symbol, and arrow markings; stop lines; crosswalk lines; speed measurement markings; parking space markings; and others.
\item Traveled Way---the portion of the roadway for the movement of vehicles, exclusive of the shoulders, berms, sidewalks, and parking lanes.
\item Turn Bay---a lane for the exclusive use of turning vehicles that is formed on the approach to the location where the turn is to be made. In most cases where turn bays are provided, drivers who desire to turn must move out of a through lane into the newly formed turn bay in order to turn. A through lane that becomes a turn lane is considered to be a dropped lane rather than a turn bay.
\item Upstream---a term that refers to a location that is encountered by traffic prior to a downstream location as it flows in an "upstream to downstream" direction. For example, "the upstream end of a lane line separating the turn lane from a through lane on the approach to an intersection" is the end of the line that is furthest from the intersection.
\item Urban Street---a type of street normally characterized by relatively low speeds, wide ranges of traffic volumes, narrower lanes, frequent intersections and driveways, significant pedestrian traffic, and more businesses and houses.
\item Vehicle---every device in, upon, or by which any person or property can be transported or drawn upon a highway, except trains and light rail transit operating in exclusive or semi-exclusive alignments. Light rail transit equipment operating in a mixed-use alignment, to which other traffic is not required to yield the right-of-way by law, is a vehicle.
\item Vibrotactile Pedestrian Device---an accessible pedestrian signal feature that communicates, by touch, information about pedestrian timing using a vibrating surface.
\item Visibility-Limited Signal Face or Visibility-Limited Signal Section---a type of signal face or signal section designed (or shielded, hooded, or louvered) to restrict the visibility of a signal indication from the side, to a certain lane or lanes, or to a certain distance from the stop line.
\item Walk Interval---an interval during which the WALKING PERSON (symbolizing WALK) signal indication is displayed.
\item Warning Beacon---a beacon used only to supplement an appropriate warning or regulatory sign or marker.
\item Warning Light---a portable, powered, yellow, lens-directed, enclosed light that is used in a temporary traffic control zone in either a steady burn or a flashing mode.
\item Warning Sign---a sign that gives notice to road users of a situation that might not be readily apparent.
\item Warrant---a warrant describes a threshold condition based upon average or normal conditions that, if found to be satisfied as part of an engineering study, shall result in analysis of other traffic conditions or factors to determine whether a traffic control device or other improvement is justified. Warrants are not a substitute for engineering judgment. The fact that a warrant for a particular traffic control device is met is not conclusive justification for the installation of the device.
\item Wayside Equipment---the signals, switches, and/or control devices for railroad or light rail transit operations housed within one or more enclosures located along the railroad or light rail transit right-of-way and/or on railroad or light rail transit property.
\item Wayside Horn System---a stationary horn (or series of horns) located at a grade crossing that is used in conjunction with train-activated or light rail transit-activated warning systems to provide audible warning of approaching rail traffic to road users on the highway or pathway approaches to a grade crossing, either as a supplement or alternative to the sounding of a locomotive horn.
\item Worker---a person on foot whose duties place him or her within the right-of-way of a street, highway, or pathway, such as street, highway, or pathway construction and maintenance forces, survey crews, utility crews, responders to incidents within the street, highway, or pathway right-of-way, and law enforcement personnel when directing traffic, investigating crashes, and handling lane closures, obstructed roadways, and disasters within the right-of-way of a street, highway, or pathway.
\item Wrong-Way Arrow---a slender, elongated, white pavement marking arrow placed upstream from the ramp terminus to indicate the correct direction of traffic flow. Wrong-way arrows are intended primarily to warn wrong-way road users that they are going in the wrong direction.
\item Yellow Change Interval---the first interval following the green or flashing arrow interval during which the steady yellow signal indication is displayed.
\item Yield Line---a row of solid white isosceles triangles pointing toward approaching vehicles extending across approach lanes to indicate the point at which the yield is intended or required to be made.
\end{enumerate}
}

\section{Meanings of Acronyms and Abbreviations in this Manual}

section

\section{Abbreviations Used on Traffic Control Devices}

section

\ifstandalone{\end{document}}