\providecommand\ifstandalone[1]{#1}
\ifstandalone{
\documentclass[9pt]{memoir}
\usepackage{graphicx}
\usepackage[hidelinks]{hyperref}
\usepackage[paperwidth=4in,paperheight=7in,top=.25in,bottom=.25in,inner=.25in,outer=.25in,includeheadfoot]{geometry}
\usepackage{enumitem}
\newcommand*{\blankpage}{%
\vspace*{\fill}
{\centering This page intentionally left blank \\ except for the presence of this message.\par}
\vspace{\fill}}
\makeatletter
\renewcommand*{\cleardoublepage}{\cleartorecto}
\renewcommand*{\cleartorecto}{\clearpage\if@twoside \ifodd\c@page\else
\blankpage
\thispagestyle{empty}
\newpage
\if@twocolumn\hbox{}\newpage\fi\fi\fi}
\renewcommand*{\cleartoverso}{\clearpage\if@twoside \ifodd\c@page
\blankpage
\thispagestyle{empty}
\newpage
\if@twocolumn\hbox{}\newpage\fi\else\fi\fi}
\makeatother

\renewcommand{\thepart}{\arabic{part}}
% \renewcommand{\thechapter} this will be defined per-book
\renewcommand{\thesection}{\thechapter.\ifnum\value{section}<10 0\fi\arabic{section}}
\renewcommand{\thefigure}{\thechapter-\arabic{figure}}
\renewcommand{\theparagraph}{\ifnum\value{paragraph}<10 0\fi\arabic{paragraph}}
\counterwithin*{chapter}{part}
\counterwithin*{part}{book}
\setlength{\cftbooknumwidth}{0pc}
\setlength{\cftbookindent}{0pc}
\renewcommand{\booknumberlinebox}[2]{#2}
\renewcommand*{\cftbookaftersnum}{:\\}
\setlength{\cftpartnumwidth}{0pc}
\setlength{\cftpartindent}{0pc}
\renewcommand{\partnumberlinebox}[2]{#2}
\renewcommand*{\cftpartname}{Part\space}
\renewcommand*{\cftpartaftersnum}{:\space}
\setlength{\cftchapternumwidth}{2pc}
\setlength{\cftsectionnumwidth}{3pc}
\setlength{\cftsectionindent}{0pc}
\renewcommand{\cftdot}{\hspace{.5pc}.\hspace{-.5pc}}
\makeatletter
\renewcommand{\@pnumwidth}{1.5pc}
\renewcommand{\@tocrmarg}{1.5pc}
\makeatother

\setlist[enumerate,1]{noitemsep,label={\Alph*.}}
\setlist[enumerate,2]{noitemsep,label={\arabic*.}}
\setlength{\emergencystretch}{3pt}

\newcommand\setnext[2]{\setcounter{#1}{#2}\addtocounter{#1}{-1}}
\newcommand\letterval[1]{%
\if#1A1%
\else\if#1B2%
\else\if#1C3%
\else\if#1D4%
\else\if#1E5%
\else\if#1F6%
\else\if#1G7%
\else\if#1H8%
\else\if#1I9%
\else\if#1J10%
\else\if#1K11%
\else\if#1L12%
\else\if#1M13%
\else\if#1N14%
\else\if#1O15%
\else\if#1P16%
\else\if#1Q17%
\else\if#1R18%
\else\if#1S19%
\else\if#1T20%
\else\if#1U21%
\else\if#1V22%
\else\if#1W23%
\else\if#1X24%
\else\if#1Y25%
\else\if#1Z26%
\else0%
\fi\fi\fi\fi\fi\fi\fi\fi\fi\fi\fi\fi\fi\fi\fi\fi\fi\fi\fi\fi\fi\fi\fi\fi\fi\fi}

\newif\ifshowstatus
\showstatustrue
\newcommand{\status}[1]{\ifshowstatus\section*{Status}\par#1\par\fi}

\renewcommand{\booknamefont}{\sffamily\huge\bfseries}
\renewcommand{\booknumfont}{\sffamily\huge\bfseries}
\renewcommand{\booktitlefont}{\sffamily\Huge\mdseries}
\renewcommand{\bookname}{}
\renewcommand{\beforebookskip}{\null\vfil\noindent\hrulefill\vfil}
%\renewcommand{\midbookskip}{\par\vskip 2\onelineskip}
\renewcommand{\afterbookskip}{\vfil\noindent\hrulefill\vfil\newpage\blankpage}
\renewcommand{\partnamefont}{\sffamily\huge\mdseries}
\renewcommand{\partnumfont}{\sffamily\huge\mdseries}
\renewcommand{\parttitlefont}{\sffamily\Huge\mdseries}
\renewcommand{\beforepartskip}{\null\vfil\noindent\hrulefill\vfil}
\renewcommand{\afterpartskip}{\vfil\vfil\newpage\blankpage}
\renewcommand{\chapnamefont}{\sffamily\huge\mdseries\itshape}
\renewcommand{\chapnumfont}{\sffamily\huge\mdseries\itshape}
\renewcommand{\chaptitlefont}{\sffamily\Huge\mdseries\upshape}
\setsecheadstyle{\sffamily\Large\mdseries}

\let\oldcontentsline\contentsline
\renewcommand\contentsline[4]{
\def\eqtesta{#1}
\ifx\eqtesta\figureliteral
\ifx\lofbook\thisbook
\oldcontentsline{#1}{#2}{#3}{#4}
\fi
\else
\oldcontentsline{#1}{#2}{#3}{#4}
\fi
}
\newcommand\startlofdomain[1]{
\write1{\string\@writefile{lof}{\string\def\string\lofbook{#1}}}
\def\thisbook{#1}
}
\def\figureliteral{figure}

\newcommand{\cfR}{23 CFR 655.603}
\newcommand{\cfra}{23 CFR 655.603(a)}
\newcommand{\cfrf}{23 Code of Federal Regulations (CFR), Part 655, Subpart F}



\renewcommand{\thechapter}{\thepart\Alph{chapter}}

\setsecnumdepth{subparagraph}
\maxsecnumdepth{subparagraph}

\newcounter{mpartype}[section]
\newcommand{\mpara}{\vspace{.5\baselineskip}\refstepcounter{paragraph}}
\newcommand{\standard}[1]{\mpara\noindent{\ifnum\thempartype=1\relax\else{}}{\bfseries{}Standard:}\\\fi\setcounter{mpartype}{1}\theparagraph\quad{\bfseries#1}}
\newcommand{\option}[1]{\mpara\noindent{}\ifnum\thempartype=2\relax\else{}Option:\\\fi\setcounter{mpartype}{2}\theparagraph\quad{#1}}
\newcommand{\support}[1]{\mpara\noindent{}\ifnum\thempartype=3\relax\else{}Support:\\\fi\setcounter{mpartype}{3}\theparagraph\quad{#1}}
\newcommand{\guidance}[1]{\mpara\noindent{}\ifnum\thempartype=4\relax\else{}{\itshape{}Guidance:}\\\fi\setcounter{mpartype}{4}\theparagraph\quad{\itshape#1}}
\newif\ifshowstatus
\showstatustrue
\newcommand{\status}[1]{\ifshowstatus\section*{Status}\par#1\par\fi}

\begin{document}
\setcounter{page}{9001}
\setnext{part}{1}
\part{General}
\setnext{chapter}{\letterval{A}}
}

\chapter{General}

\status{Check for things that need to be changed for Centralia setting in first two sections; fill in remaining sections.}

\section{Purpose of Traffic Control Devices}

\support{
The purpose of traffic control devices, as well as the principles for their use, is to promote highway safety and efficiency by providing for the orderly movement of all road users on streets, highways, bikeways, and private roads open to public travel throughout the Nation.}

\support{
Traffic control devices notify road users of regulations and provide warning and guidance needed for the uniform and efficient operation of all elements of the traffic stream in a manner intended to minimize the occurrences of crashes.}

\standard{
Traffic control devices or their supports shall not bear any advertising message or any other message that is not related to traffic control.}

\support{
Tourist-oriented directional signs and Specific Service signs are not considered advertising; rather, they are classified as motorist service signs.}

\section{Principles of Traffic Control Devices}

\support{
This Manual contains the basic principles that govern the design and use of traffic control devices for all streets, highways, bikeways, and private roads open to public travel (see definition in Section~\ref{sec:2009.1A.13} on page~\pageref{sec:2009.1A.13}) regardless of type or class or the public agency, official, or owner having jurisdiction. This Manual's text specifies the restriction on the use of a device if it is intended for limited application or for a specific system. It is important that these principles be given primary consideration in the selection and application of each device.}

\guidance{
To be effective, a traffic control device should meet five basic requirements:
\begin{enumerate}
\item Fulfill a need;
\item Command attention;
\item Convey a clear, simple meaning;
\item Command respect from road users; and
\item Give adequate time for proper response.
\end{enumerate}}

\guidance{
Design, placement, operation, maintenance, and uniformity are aspects that should be carefully considered in order to maximize the ability of a traffic control device to meet the five requirements listed in the previous paragraph. Vehicle speed should be carefully considered as an element that governs the design, operation, placement, and location of various traffic control devices.}

\support{
The definition of the word ``speed'' varies depending on its use. The definitions of specific speed terms are contained in Section~\ref{sec:2009.1A.13} on page~\pageref{sec:2009.1A.13}.}

\guidance{
The actions required of road users to obey regulatory devices should be specified by State statute, or in cases not covered by State statute, by local ordinance or resolution. Such statutes, ordinances, and resolutions should be consistent with the ``Uniform Vehicle Code'' (see Section~\ref{sec:2009.1A.11} on page~\pageref{sec:2009.1A.11}).}

\guidance{
The proper use of traffic control devices should provide the reasonable and prudent road user with the information necessary to efficiently and lawfully use the streets, highways, pedestrian facilities, and bikeways.}

\support{
Uniformity of the meaning of traffic control devices is vital to their effectiveness. The meanings ascribed to devices in this Manual are in general accord with the publications mentioned in Section~\ref{sec:2009.1A.11} on page~\pageref{sec:2009.1A.11}.}

\section{Design of Traffic Control Devices}

section

\section{Placement and Operation of Traffic Control Devices}

section

\section{Maintenance of Traffic Control Devices}

section

\section{Uniformity of Traffic Control Devices}

section

\section{Responsibility for Traffic Control Devices}

section

\section{Authority for Placement of Traffic Control Devices}

section

\section{Engineering Study and Engineering Judgment}

section

\section{Interpretations, Experimentations, Changes, and Interim Approvals}

section

\section{Relation to Other Publications}
\label{sec:2009.1A.11}

section

\section{Color Code}

section

\section{Definitions of Headings, Words, and Phrases in this Manual}
\label{sec:2009.1A.13}
\status{Fill in remainder of this section.}

\standard{
\label{p:2009.1A.13.01}
When used in this Manual, the text headings of Standard, Guidance, Option, and Support shall be defined as follows:
\begin{enumerate}
\item Standard---a statement of required, mandatory, or specifically prohibitive practice regarding a traffic control device. All Standard statements are labeled, and the text appears in bold type. The verb ``shall'' is typically used. The verbs ``should'' and ``may'' are not used in Standard statements. Standard statements are sometimes modified by Options.
\item Guidance---a statement of recommended, but not mandatory, practice in typical situations, with deviations allowed if engineering judgment or engineering study indicates the deviation to be appropriate. All Guidance statements are labeled, and the text appears in unbold type. The verb ``should'' is typically used. The verbs ``shall'' and ``may'' are not used in Guidance statements. Guidance statements are sometimes modified by Options.
\item Option---a statement of practice that is a permissive condition and carries no requirement or recommendation. Option statements sometime contain allowable modifications to a Standard or Guidance statement. All Option statements are labeled, and the text appears in unbold type. The verb ``may'' is typically used. The verbs ``shall'' and ``should'' are not used in Option statements.
\item Support---an informational statement that does not convey any degree of mandate, recommendation, authorization, prohibition, or enforceable condition. Support statements are labeled, and the text appears in unbold type. The verbs ``shall,'' ``should,'' and ``may'' are not used in Support statements.
\end{enumerate}}


\section{Meanings of Acronyms and Abbreviations in this Manual}

section

\section{Abbreviations Used on Traffic Control Devices}

section

\ifstandalone{\end{document}}