\providecommand\ifstandalone[1]{#1}
\ifstandalone{
\documentclass[9pt]{memoir}
\usepackage{graphicx}
\usepackage[hidelinks]{hyperref}
\usepackage[paperwidth=4in,paperheight=7in,top=.25in,bottom=.25in,inner=.25in,outer=.25in,includeheadfoot]{geometry}
\usepackage{enumitem}
\usepackage[all]{xypic}

\newcommand*{\blankpage}{%
\vspace*{\fill}
{\centering This page intentionally left blank \\ except for the presence of this message.\par}
\vspace{\fill}}
\makeatletter
\renewcommand*{\cleardoublepage}{\cleartorecto}
\renewcommand*{\cleartorecto}{\clearpage\if@twoside \ifodd\c@page\else
\blankpage
\thispagestyle{empty}
\newpage
\if@twocolumn\hbox{}\newpage\fi\fi\fi}
\renewcommand*{\cleartoverso}{\clearpage\if@twoside \ifodd\c@page
\blankpage
\thispagestyle{empty}
\newpage
\if@twocolumn\hbox{}\newpage\fi\else\fi\fi}
\makeatother

\renewcommand{\thepart}{\arabic{part}}
% \renewcommand{\thechapter} this will be defined per-book
\renewcommand{\thesection}{\thechapter.\ifnum\value{section}<10 0\fi\arabic{section}}
\renewcommand{\thefigure}{\thechapter-\arabic{figure}}
\renewcommand{\theparagraph}{\ifnum\value{paragraph}<10 0\fi\arabic{paragraph}}
\counterwithin*{chapter}{part}
\counterwithin*{part}{book}
\setlength{\cftbooknumwidth}{0pc}
\setlength{\cftbookindent}{0pc}
\renewcommand{\booknumberlinebox}[2]{#2}
\renewcommand*{\cftbookaftersnum}{:\\}
\setlength{\cftpartnumwidth}{0pc}
\setlength{\cftpartindent}{0pc}
\renewcommand{\partnumberlinebox}[2]{#2}
\renewcommand*{\cftpartname}{Part\space}
\renewcommand*{\cftpartaftersnum}{:\space}
\setlength{\cftchapternumwidth}{2pc}
\setlength{\cftsectionnumwidth}{3pc}
\setlength{\cftsectionindent}{0pc}
\renewcommand{\cftdot}{\hspace{.5pc}.\hspace{-.5pc}}
\makeatletter
\renewcommand{\@pnumwidth}{1.5pc}
\renewcommand{\@tocrmarg}{1.5pc}
\makeatother

\setlist[enumerate,1]{noitemsep,label={\Alph*.}}
\setlist[enumerate,2]{noitemsep,label={\arabic*.}}
\setlength{\emergencystretch}{3pt}

\newcommand\setnext[2]{\setcounter{#1}{#2}\addtocounter{#1}{-1}}
\newcommand\letterval[1]{%
\if#1A1%
\else\if#1B2%
\else\if#1C3%
\else\if#1D4%
\else\if#1E5%
\else\if#1F6%
\else\if#1G7%
\else\if#1H8%
\else\if#1I9%
\else\if#1J10%
\else\if#1K11%
\else\if#1L12%
\else\if#1M13%
\else\if#1N14%
\else\if#1O15%
\else\if#1P16%
\else\if#1Q17%
\else\if#1R18%
\else\if#1S19%
\else\if#1T20%
\else\if#1U21%
\else\if#1V22%
\else\if#1W23%
\else\if#1X24%
\else\if#1Y25%
\else\if#1Z26%
\else0%
\fi\fi\fi\fi\fi\fi\fi\fi\fi\fi\fi\fi\fi\fi\fi\fi\fi\fi\fi\fi\fi\fi\fi\fi\fi\fi}

\newif\ifshowstatus
\showstatustrue
\newcommand{\status}[1]{\ifshowstatus\section*{Status}\par#1\par\fi}

\renewcommand{\booknamefont}{\sffamily\huge\bfseries}
\renewcommand{\booknumfont}{\sffamily\huge\bfseries}
\renewcommand{\booktitlefont}{\sffamily\Huge\mdseries}
\renewcommand{\bookname}{}
\renewcommand{\beforebookskip}{\null\vfil\noindent\hrulefill\vfil}
%\renewcommand{\midbookskip}{\par\vskip 2\onelineskip}
\renewcommand{\afterbookskip}{\vfil\noindent\hrulefill\vfil\newpage\blankpage}
\renewcommand{\partnamefont}{\sffamily\huge\mdseries}
\renewcommand{\partnumfont}{\sffamily\huge\mdseries}
\renewcommand{\parttitlefont}{\sffamily\Huge\mdseries}
\renewcommand{\beforepartskip}{\null\vfil\noindent\hrulefill\vfil}
\renewcommand{\afterpartskip}{\vfil\vfil\newpage\blankpage}
\renewcommand{\chapnamefont}{\sffamily\huge\mdseries\itshape}
\renewcommand{\chapnumfont}{\sffamily\huge\mdseries\itshape}
\renewcommand{\chaptitlefont}{\sffamily\Huge\mdseries\upshape}
\setsecheadstyle{\sffamily\Large\mdseries}

\let\oldcontentsline\contentsline
\renewcommand\contentsline[4]{
\def\eqtesta{#1}
\ifx\eqtesta\figureliteral
\ifx\lofbook\thisbook
\oldcontentsline{#1}{#2}{#3}{#4}
\fi
\else
\oldcontentsline{#1}{#2}{#3}{#4}
\fi
}
\newcommand\startlofdomain[1]{
\write1{\string\@writefile{lof}{\string\def\string\lofbook{#1}}}
\def\thisbook{#1}
}
\def\figureliteral{figure}

\newcommand{\cfR}{23 CFR 655.603}
\newcommand{\cfra}{23 CFR 655.603(a)}
\newcommand{\cfrf}{23 Code of Federal Regulations (CFR), Part 655, Subpart F}



\renewcommand{\thechapter}{\thepart\Alph{chapter}}

\setsecnumdepth{subparagraph}
\maxsecnumdepth{subparagraph}

\newcounter{mpartype}[section]
\newcommand{\mpara}{\vspace{.5\baselineskip}\refstepcounter{paragraph}}
\newcommand{\standard}[1]{\mpara\noindent{\ifnum\thempartype=1\relax\else{}}{\bfseries{}Standard:}\\\fi\setcounter{mpartype}{1}\theparagraph\quad{\bfseries#1}}
\newcommand{\option}[1]{\mpara\noindent{}\ifnum\thempartype=2\relax\else{}Option:\\\fi\setcounter{mpartype}{2}\theparagraph\quad{#1}}
\newcommand{\support}[1]{\mpara\noindent{}\ifnum\thempartype=3\relax\else{}Support:\\\fi\setcounter{mpartype}{3}\theparagraph\quad{#1}}
\newcommand{\guidance}[1]{\mpara\noindent{}\ifnum\thempartype=4\relax\else{}{\itshape{}Guidance:}\\\fi\setcounter{mpartype}{4}\theparagraph\quad{\itshape#1}}
\newif\ifshowstatus
\showstatustrue
\newcommand{\status}[1]{\ifshowstatus\section*{Status}\par#1\par\fi}

\begin{document}
\setcounter{page}{9001}
\setnext{part}{1}
\part{General}
\setnext{chapter}{\letterval{A}}
}

\chapter{General}

\status{Check for things that need to be changed for Centralia setting in first two sections; fill in remaining sections.}

\section{Purpose of Traffic Control Devices}

\support{
The purpose of traffic control devices, as well as the principles for their use, is to promote highway safety and efficiency by providing for the orderly movement of all road users on streets, highways, bikeways, and private roads open to public travel throughout the Nation.}

\support{
Traffic control devices notify road users of regulations and provide warning and guidance needed for the uniform and efficient operation of all elements of the traffic stream in a manner intended to minimize the occurrences of crashes.}

\standard{
Traffic control devices or their supports shall not bear any advertising message or any other message that is not related to traffic control.}

\support{
Tourist-oriented directional signs and Specific Service signs are not considered advertising; rather, they are classified as motorist service signs.}

\section{Principles of Traffic Control Devices}

\support{
This Manual contains the basic principles that govern the design and use of traffic control devices for all streets, highways, bikeways, and private roads open to public travel (see definition in Section~\ref{sec:2009.1A.13} on page~\pageref{sec:2009.1A.13}) regardless of type or class or the public agency, official, or owner having jurisdiction. This Manual's text specifies the restriction on the use of a device if it is intended for limited application or for a specific system. It is important that these principles be given primary consideration in the selection and application of each device.}

\guidance{
To be effective, a traffic control device should meet five basic requirements:
\begin{enumerate}
\item Fulfill a need;
\item Command attention;
\item Convey a clear, simple meaning;
\item Command respect from road users; and
\item Give adequate time for proper response.
\end{enumerate}}

\guidance{
Design, placement, operation, maintenance, and uniformity are aspects that should be carefully considered in order to maximize the ability of a traffic control device to meet the five requirements listed in the previous paragraph. Vehicle speed should be carefully considered as an element that governs the design, operation, placement, and location of various traffic control devices.}

\support{
The definition of the word ``speed'' varies depending on its use. The definitions of specific speed terms are contained in Section~\ref{sec:2009.1A.13} on page~\pageref{sec:2009.1A.13}.}

\guidance{
The actions required of road users to obey regulatory devices should be specified by State statute, or in cases not covered by State statute, by local ordinance or resolution. Such statutes, ordinances, and resolutions should be consistent with the ``Uniform Vehicle Code'' (see Section~\ref{sec:2009.1A.11} on page~\pageref{sec:2009.1A.11}).}

\guidance{
The proper use of traffic control devices should provide the reasonable and prudent road user with the information necessary to efficiently and lawfully use the streets, highways, pedestrian facilities, and bikeways.}

\support{
Uniformity of the meaning of traffic control devices is vital to their effectiveness. The meanings ascribed to devices in this Manual are in general accord with the publications mentioned in Section~\ref{sec:2009.1A.11} on page~\pageref{sec:2009.1A.11}.}

\section{Design of Traffic Control Devices}

\guidance{
Devices should be designed so that features such as size, shape, color, composition, lighting or retroreflection, and contrast are combined to draw attention to the devices; that size, shape, color, and simplicity of message combine to produce a clear meaning; that legibility and size combine with placement to permit adequate time for response; and that uniformity, size, legibility, and reasonableness of the message combine to command respect.}

\guidance{
Aspects of a device's standard design should be modified only if there is a demonstrated need.}

\support{
An example of modifying a device's design would be to modify the Combination Horizontal Alignment/Intersection (W1-10) sign to show intersecting side roads on both sides rather than on just one side of the major road within the curve.}

\option{
With the exception of symbols and colors, minor modifications in the specific design elements of a device may be made provided the essential appearance characteristics are preserved.}

\section{Placement and Operation of Traffic Control Devices}

\guidance{
Placement of a traffic control device should be within the road user's view so that adequate visibility is provided. To aid in conveying the proper meaning, the traffic control device should be appropriately positioned with respect to the location, object, or situation to which it applies. The location and legibility of the traffic control device should be such that a road user has adequate time to make the proper response in both day and night conditions.}

\guidance{
Traffic control devices should be placed and operated in a uniform and consistent manner.}

\guidance{
Unnecessary traffic control devices should be removed. The fact that a device is in good physical condition should not be a basis for deferring needed removal or change.}

\section{Maintenance of Traffic Control Devices}

\guidance{
Functional maintenance of traffic control devices should be used to determine if certain devices need to be changed to meet current traffic conditions.}

\guidance{
Physical maintenance of traffic control devices should be performed to retain the legibility and visibility of the device, and to retain the proper functioning of the device.}

\support{
Clean, legible, properly mounted devices in good working condition command the respect of road users.}

\section{Uniformity of Traffic Control Devices}

\support{
Uniformity of devices simplifies the task of the road user because it aids in recognition and understanding, thereby reducing perception/reaction time. Uniformity assists road users, law enforcement officers, and traffic courts by giving everyone the same interpretation. Uniformity assists public highway officials through efficiency in manufacture, installation, maintenance, and administration. Uniformity means treating similar situations in a similar way. The use of uniform traffic control devices does not, in itself, constitute uniformity. A standard device used where it is not appropriate is as objectionable as a non-standard device; in fact, this might be worse, because such misuse might result in disrespect at those locations where the device is needed and appropriate.}

\section{Responsibility for Traffic Control Devices}

\standard{
The responsibility for the design, placement, operation, maintenance, and uniformity of traffic control devices shall rest with the public agency or the official having jurisdiction, or, in the case of private roads open to public travel, with the private owner or private official having jurisdiction. \cfR{} adopts the MUTCD as the national standard for all traffic control devices installed on any street, highway, bikeway, or private road open to public travel  (see definition in Section~\ref{sec:2009.1A.13} on page~\pageref{sec:2009.1A.13}). When a State or other Federal agency manual or supplement is required, that manual or supplement shall be in substantial conformance with the National MUTCD.}

\standard{
\cfR{} also states that traffic control devices on all streets, highways, bikeways, and private roads open to public travel in each State shall be in substantial conformance with standards issued or endorsed by the Federal Highway Administrator.}

\support{
The Introduction of this Manual contains information regarding the meaning of substantial conformance and the applicability of the MUTCD to private roads open to public travel.}

\support{
The ``Uniform Vehicle Code'' (see Section~\ref{sec:2009.1A.11} on page~\pageref{sec:2009.1A.11}) has the following provision in Section 15-104 for the adoption of a uniform manual:

\begin{quote}
    ``The [State Highway Agency] shall adopt a manual and specification for a uniform system of traffic control devices consistent with the provisions of this code for use upon highways within this State. Such uniform system shall correlate with and so far as possible conform to the system set forth in the most recent edition of the Manual on Uniform Traffic Control Devices for Streets and Highways, and other standards issued or endorsed by the Federal Highway Administrator.''
    
    ``The Manual adopted pursuant to subsection (a) shall have the force and effect of law.''
\end{quote}}

\support{
All States have officially adopted the National MUTCD either in its entirety, with supplemental provisions, or as a separate published document.}

\guidance{
These individual State manuals or supplements should be reviewed for specific provisions relating to that State.}

\support{
The National MUTCD has also been adopted by the National Park Service, the U.S. Forest Service, the U.S. Military Command, the Bureau of Indian Affairs, the Bureau of Land Management, and the U.S. Fish and Wildlife Service.}

\guidance{
States should adopt Section 15-116 of the ``Uniform Vehicle Code,'' which states that, ``No person shall install or maintain in any area of private property used by the public any sign, signal, marking, or other device intended to regulate, warn, or guide traffic unless it conforms with the State manual and specifications adopted under Section 15-104.''}

\section{Authority for Placement of Traffic Control Devices}

\standard{
Traffic control devices, advertisements, announcements, and other signs or messages within the highway right-of-way shall be placed only as authorized by a public authority or the official having jurisdiction, or, in the case of private roads open to public travel, by the private owner or private official having jurisdiction, for the purpose of regulating, warning, or guiding traffic.}

\standard{
When the public agency or the official having jurisdiction over a street or highway or, in the case of private roads open to public travel, the private owner or private official having jurisdiction, has granted proper authority, others such as contractors and public utility companies shall be permitted to install temporary traffic control devices in temporary traffic control zones. Such traffic control devices shall conform with the Standards of this Manual.}

\standard{
All regulatory traffic control devices shall be supported by laws, ordinances, or regulations.}

\support{
Provisions of this Manual are based upon the concept that effective traffic control depends upon both appropriate application of the devices and reasonable enforcement of the regulations.}

\support{
Although some highway design features, such as curbs, median barriers, guardrails, speed humps or tables, and textured pavement, have a significant impact on traffic operations and safety, they are not considered to be traffic control devices and provisions regarding their design and use are generally not included in this Manual.}

\support{
Certain types of signs and other devices that do not have any traffic control purpose are sometimes placed within the highway right-of-way by or with the permission of the public agency or the official having jurisdiction over the street or highway. Most of these signs and other devices are not intended for use by road users in general, and their message is only important to individuals who have been instructed in their meanings. These signs and other devices are not considered to be traffic control devices and provisions regarding their design and use are not included in this Manual. Among these signs and other devices are the following:

\begin{enumerate}
   \item Devices whose purpose is to assist highway maintenance personnel. Examples include markers to guide snowplow operators, devices that identify culvert and drop inlet locations, and devices that precisely identify highway locations for maintenance or mowing purposes.
   \item Devices whose purpose is to assist fire or law enforcement personnel. Examples include markers that identify fire hydrant locations, signs that identify fire or water district boundaries, speed measurement pavement markings, small indicator lights to assist in enforcement of red light violations, and photo enforcement systems.
   \item Devices whose purpose is to assist utility company personnel and highway contractors, such as markers that identify underground utility locations.
   \item Signs posting local non-traffic ordinances.
   \item Signs giving civic organization meeting information.
\end{enumerate}}

\standard{
Signs and other devices that do not have any traffic control purpose that are placed within the highway right-of-way shall not be located where they will interfere with, or detract from, traffic control devices.}

\guidance{
Any unauthorized traffic control device or other sign or message placed on the highway right-of-way by a private organization or individual constitutes a public nuisance and should be removed. All unofficial or non-essential traffic control devices, signs, or messages should be removed.}

\section{Engineering Study and Engineering Judgment}

\support{
Definitions of an engineering study and engineering judgment are contained in in Section~\ref{sec:2009.1A.13} on page~\pageref{sec:2009.1A.13}.}

\standard{
This Manual describes the application of traffic control devices, but shall not be a legal requirement for their installation.}

\guidance{
The decision to use a particular device at a particular location should be made on the basis of either an engineering study or the application of engineering judgment. Thus, while this Manual provides Standards, Guidance, and Options for design and applications of traffic control devices, this Manual should not be considered a substitute for engineering judgment. Engineering judgment should be exercised in the selection and application of traffic control devices, as well as in the location and design of roads and streets that the devices complement.}

\guidance{
Early in the processes of location and design of roads and streets, engineers should coordinate such location and design with the design and placement of the traffic control devices to be used with such roads and streets.}

\guidance{
Jurisdictions, or owners of private roads open to public travel, with responsibility for traffic control that do not have engineers on their staffs who are trained and/or experienced in traffic control devices should seek engineering assistance from others, such as the State transportation agency, their county, a nearby large city, or a traffic engineering consultant.}

\support{
As part of the Federal-aid Program, each State is required to have a Local Technology Assistance Program (LTAP) and to provide technical assistance to local highway agencies. Requisite technical training in the application of the principles of the MUTCD is available from the State's Local Technology Assistance Program for needed engineering guidance and assistance.}

\section{Interpretations, Experimentations, Changes, and Interim Approvals}

\standard{
Design, application, and placement of traffic control devices other than those adopted in this Manual shall be prohibited unless the provisions of this Section are followed.}

\support{
Continuing advances in technology will produce changes in the highway, vehicle, and road user proficiency; therefore, portions of the system of traffic control devices in this Manual will require updating. In addition, unique situations often arise for device applications that might require interpretation or clarification of this Manual. It is important to have a procedure for recognizing these developments and for introducing new ideas and modifications into the system.}

\standard{
Except as provided in Paragraph \ref{p:2009.a1.10p4}, requests for any interpretation, permission to experiment, interim approval, or change shall be submitted electronically to the Federal Highway Administration (FHWA), Office of Transportation Operations, MUTCD team, at the following e-mail address: \href{mailto:MUTCDofficialrequest@dot.gov}{\nolinkurl{MUTCDofficialrequest@dot.gov}}.}

\option{
\label{p:2009.a1.10p4}
If electronic submittal is not possible, requests for interpretations, permission to experiment, interim approvals, or changes may instead be mailed to the Office of Transportation Operations, HOTO-1, Federal Highway Administration, 1200 New Jersey Avenue, SE, Washington, DC 20590.}

\support{
Communications regarding other MUTCD matters that are not related to official requests will receive quicker attention if they are submitted electronically to the MUTCD Team Leader or to the appropriate individual MUTCD team member. Their e-mail addresses are available through the links contained on the "Who's Who" page on the MUTCD website at \url{http://mutcd.fhwa.dot.gov/team.htm}.}

\support{
An interpretation includes a consideration of the application and operation of standard traffic control devices, official meanings of standard traffic control devices, or the variations from standard device designs.}

\guidance{
Requests for an interpretation of this Manual should contain the following information:

\begin{enumerate}
   \item A concise statement of the interpretation being sought;
   \item A description of the condition that provoked the need for an interpretation;
   \item Any illustration that would be helpful to understand the request; and
   \item Any supporting research data that is pertinent to the item to be interpreted.
\end{enumerate}}

\support{
Requests to experiment include consideration of field deployment for the purpose of testing or evaluating a new traffic control device, its application or manner of use, or a provision not specifically described in this Manual.}

\support{
A request for permission to experiment will be considered only when submitted by the public agency or troll facility operator responsible for the operation of the road or street on which the experiment is to take place. For a private road open to public travel, the request will be considered only if it is submitted by the private owner or private official having jurisdiction.}

\support{
A diagram indicating the process for experimenting with traffic control devices is shown in Figure \ref{fig:2009.a1-1} on page \pageref{fig:2009.a1-1}.}

\begin{figure}[p]
\caption{Process for Requesting and Conducting Experimentations for New Traffic Control Devices}
\label{fig:2009.a1-1}
\makebox[\textwidth]{\small
$$\xymatrix@=9pt{
& *+[F]{\txt{Requesting\\jurisdiction\\submits request\\to FHWA}}\ar[d] \\
%
& *+[F]{\txt{FHWA Review}}\ar[d] & \\
%
& *++[o][F]{\txt{Approved?}}\ar[r]^<<<{\txt{\tiny NO}}\ar[d]^<<<{\txt{\tiny YES}} 
& *+[F]{\txt{Requesting\\jurisdiction\\responds to\\questions raised\\by FHWA}}\ar`u[ul][ul] \\
%
& *+[F]{\txt{Requesting\\jurisdiction\\installs\\experimental\\traffic control\\device}}\ar[dl]\ar[dr] \\
%
*+[F]{\txt{Evaluate\\experimental\\traffic control\\device}}\ar@{-->}[rr]\ar[dr]
& & *+[F]{\txt{Requesting\\jurisdiction\\provides\\semi-annual\\reports to\\FHWA Division\\\& HQ}}\ar[dl] \\
%
& *+[F]{\txt{Requesting\\jurisdiction\\provides\\FHWA a copy\\of final report}}
}$$}\par
\end{figure}

\guidance{
The request for permission to experiment should contain the following:

\begin{enumerate}
   \item A statement indicating the nature of the problem.
   \item A description of the proposed change to the traffic control device or application of the traffic control device, how it was developed, the manner in which it deviates from the standard, and how it is expected to be an improvement over existing standards.
   \item Any illustration that would be helpful to understand the traffic control device or use of the traffic control device.
   \item Any supporting data explaining how the traffic control device was developed, if it has been tried, in what ways it was found to be adequate or inadequate, and how this choice of device or application was derived.
   \item A legally binding statement certifying that the concept of the traffic control device is not protected by a patent or copyright. (An example of a traffic control device concept would be countdown pedestrian signals in general. Ordinarily an entire general concept would not be patented or copyrighted, but if it were it would not be acceptable for experimentation unless the patent or copyright owner signs a waiver of rights acceptable to the FHWA. An example of a patented or copyrighted specific device within the general concept of countdown pedestrian signals would be a manufacturer's design for its specific brand of countdown signal, including the design details of the housing or electronics that are unique to that manufacturer's product. As long as the general concept is not patented or copyrighted, it is acceptable for experimentation to incorporate the use of one or more patented devices of one or several manufacturers.)
   \item The time period and location(s) of the experiment.
   \item A detailed research or evaluation plan that must provide for close monitoring of the experimentation, especially in the early stages of its field implementation. The evaluation plan should include before and after studies as well as quantitative data describing the performance of the experimental device.
   \item An agreement to restore the site of the experiment to a condition that complies with the provisions of this Manual within 3 months following the end of the time period of the experiment. This agreement must also provide that the agency sponsoring the experimentation will terminate the experimentation at any time that it determines significant safety concerns are directly or indirectly attributable to the experimentation. The FHWA's Office of Transportation Operations has the right to terminate approval of the experimentation at any time if there is an indication of safety concerns. If, as a result of the experimentation, a request is made that this Manual be changed to include the device or application being experimented with, the device or application will be permitted to remain in place until an official rulemaking action has occurred.
   \item An agreement to provide semi-annual progress reports for the duration of the experimentation, and an agreement to provide a copy of the final results of the experimentation to the FHWA's Office of Transportation Operations within 3 months following completion of the experimentation. The FHWA's Office of Transportation Operations has the right to terminate approval of the experimentation if reports are not provided in accordance with this schedule.
\end{enumerate}}

\support{
A change includes consideration of a new device to replace a present standard device, an additional device to be added to the list of standard devices, or a revision to a traffic control device application or placement criteria.}

\guidance{
Requests for a change to this Manual should contain the following information:

\begin{enumerate}
   \item A statement indicating what change is proposed;
   \item Any illustration that would be helpful to understand the request; and
   \item Any supporting research data that is pertinent to the item to be reviewed.
\end{enumerate}}

\support{
Interim approval allows interim use, pending official rulemaking, of a new traffic control device, a revision to the application or manner of use of an existing traffic control device, or a provision not specifically described in this Manual. The FHWA issues an Interim Approval by official memorandum signed by the Associate Administrator for Operations and posts this memorandum on the MUTCD website. the issuance by FHWA of an interim approval will typically result in the traffic control device or application being placed into the next scheduled rulemaking process for revisions to this Manual.}

\support{
Interim approval is considered based on the results of successful experimentation, results of analytical or laboratory studies, and/or review of non-U.S. experience with a traffic control device or application. Interim approval considerations include an assessment of relative risks, benefits, costs, impacts, and other factors.}

\support{
Interim approval allows for optional use of a traffic control device or application and does not create a new mandate or recommendation for use. Interim approval includes conditions that jurisdictions agree to comply with in order to use the traffic control device or application until an official rulemaking action has occurred.}

\standard{
A jurisdiction, troll facility operator, or owner of a private road open to public travel that desires to use a traffic control device for which FHWA has issued an interim approval shall request permission from FHWA.}

\guidance{
\label{p:2009.1a.11p18}
The request for permission to place a traffic control device under an interim approval should contain the following:

\begin{enumerate}
   \item A description of where the device will be used, such as a list of specific locations or highway segments or types of situations, or a statement of the intent to use the device jurisdiction-wide;
   \item An agreement to abide by the specific conditions for use of the device as contained in the FHWA's interim approval document;
   \item An agreement to maintain and continually update a list of locations where the device has been installed; and
   \item An agreement to:
   \begin{enumerate}
       \item Restore the site(s) of the interim approval to a condition that complies with the provisions in this Manual within 3 months following the issuance of a final rule on this traffic control device; and
       \item Terminate use of the device or application installed under the interim approval at any time that it determines significant safety concerns are directly or indirectly attributable to the device or application. The FHWA's Office of Transportation Operations has the right to terminate the interim approval at any time if there is an indication of safety concerns.
\end{enumerate}
\end{enumerate}}

\option{
A State may submit a request for the use of a device under interim approval for all jurisdictions in that State, as long as the request contains the information listed in Paragraph \ref{p:2009.1a.11p18}.}

\guidance{
A local jurisdiction, troll facility operator, or owner of a private road open to public travel using a traffic control device or application under an interim approval that was granted by FHWA either directly or on a statewide basis based on the State's request should inform the State of the locations of such use.}

\guidance{
A local jurisdiction, troll facility operator, or owner of a private road open to public travel that is requesting permission to experiment or permission to use a device or application under an interim approval should first check for any State laws and/or directives covering the application of the MUTCD provisions that might exist in their State.}

\option{
A device or application installed under an interim approval may remain in place, under the conditions established in the interim approval, until an official rulemaking action has occurred.}

\support{
A diagram indicating the process for incorporating new traffic control devices into this Manual is shown in Figure \ref{fig:2009.a1-2} on page \pageref{fig:2009.a1-2}.}

\begin{figure}[p]
\caption{Process for Incorporating New Traffic Control Devices into the MUTCD}
\label{fig:2009.a1-2}
\makebox[\textwidth]{\scriptsize
$$\xymatrix@C=2pt@R=7pt{
*+[F]{\txt{Analytical or\\laboratory study\\results and/or\\non-Centralia\\experimentation}}\ar[dr]
& *+[F]{\txt{Experiment\\successful (see\\Figure \ref{fig:2009.a1-1})}}\ar[d]
& *+[F]{\txt{Request for\\change from\\jurisdiction or\\interested party}}\ar[dl] \\
%
& *+[F]{\txt{FHWA\\review}}\ar`r[dr][dr] & \\
%
*+[F]{\txt{Jurisdiction\\restores\\experiment\\site to\\original\\condition}}
& *++[o][F]{\txt{Further\\experimentation\\required?}}\ar[l]_<<<{\txt{\tiny NO}}\ar`d[dl]^<<<{\txt{\tiny YES}}[dl]
& *++[o][F]{\txt{Accepted\\for Federal\\rulemaking?}}\ar[l]_<<<{\txt{\tiny NO}}\ar[dd]^<<<{\txt{\tiny YES}}\ar`d[dl]`l[ddl][ddl] \\
%
*+[F]{\txt{See Figure\\\ref{fig:2009.a1-1}}}
& & \\
%
*+[F]{\txt{FHWA notifies\\interested parties\\(if any)}}
& *++[o][F]{\txt{Interim\\approval?}}\ar[l]_<<<{\txt{\tiny NO}}\ar[d]^<<{\txt{\tiny YES}}
& *+[F]{\txt{FHWA prepares\\Notice of Proposed\\Amendment}}\ar[d] \\
%
& *+[F]{\txt{FHWA issues Interim\\Approval with technical\\conditions for use, and\\posts on MUTCD website}}\ar[d]
& *+[F]{\txt{FHWA publishes\\Notice of Proposed\\Amendment in\\Federal Register}}\ar[d] \\
%
& *+[F]{\txt{Jurisdictions apply\\for and receive\\Interim Approval}}\ar[d]
& *+[F]{\txt{Docket comment\\period}}\ar[d] \\
%
& *+[F]{\txt{Jurisdictions deploy\\devices under Interim\\Approval conditions}}\ar[dd]
& *+[F]{\txt{FHWA reviews\\comments}}\ar[d] \\
%
& & *+[F]{\txt{FHWA prepares\\Final Rule}}\ar[d] \\
%
*+[F]{\txt{No action\\required}}
& *++[o][F]{\txt{Final\\Rule different\\from Interim\\Approval?}}\ar[l]_<<<{\txt{\tiny NO}}\ar[d]^<<<{\txt{\tiny YES}}
& *+[F]{\txt{FHWA publishes\\Final Rule}}\ar[l]\ar[d] \\
%
& *+[F]{\txt{Jurisdictions restore sites\\of Interim Approval to\\previous condition and/or\\comply with Final Rule}}
& *+[F]{\txt{State Manuals must\\be in substantial\\conformance with the\\National MUTCD within\\2 years as specified\\in \cfra}}
}$$}\par
\end{figure}

\support{
For additional information concerning interpretations, experimentation, changes, or interim approvals, visit the MUTCD website at \url{http://mutcd.fhwa.dot.gov}.}

\section{Relation to Other Publications}
\label{sec:2009.1A.11}

\status{Lots of attention needed here for Centralianizing.}

\standard{
\label{p:2009.A1.11p01}
To the extent that they are incorporated by specific reference, the latest editions of the following publications, or those editions specifically noted, shall be a part of this Manual: ``Royal Centralia Sign Design Manual'' book (FHWA); and ``Royal Centralia Comprendium of Sign Fabrication Standards'' (appendix to subpart F of Part 655 of Title 23 of the Code of Federal Regulations).}

\support{
The ``Royal Centralia Sign Design Manual'' book includes standard alphabets and symbols and arrows for signs and pavement markings.}

\support{
For information about the publications mentioned in Paragraph \ref{p:2009.A1.11p01}, visit the Federal Highway Administration's MUTCD website at http://mutcd.fhwa.dot.gov, or write to the FHWA, 1200 New Jersey Avenue, SE, HOTO, Washington, DC 20590.}

\support{
Other publications that are useful sources of information with respect to the use of this Manual are listed in this paragraph. See Addresses in this Manual for ordering information for the following publications (later editions might also be available as useful sources of information):

\begin{itemize}
\item    ``AAA School Safety Patrol Operations Manual,'' 2006 Edition (American Automobile Association---AAA)
\item    ``A Policy on Geometric Design of Highways and Streets,'' 2004 Edition (American Association of State Highway and Transportation Officials---AASHTO)
\item    ``Guide for the Development of Bicycle Facilities,'' 1999 Edition (AASHTO)
\item    ``Guide for the Planning, Design, and Operation of Pedestrian Facilities,'' 2004 Edition (AASHTO)
\item    ``Guide to Metric Conversion,'' 1993 Edition (AASHTO)
\item    ``Guidelines for the Selection of Supplemental Guide Signs for Traffic Generators Adjacent to Freeways,'' 4th Edition/Guide Signs, Part II: Guidelines for Airport Guide Signing/Guide Signs, Part III: List of Control Cities for Use in Guide Signs on Interstate Highways,'' Item Code: GSGLC-4, 2001 Edition (AASHTO)
\item    ``Roadside Design Guide,'' 2006 Edition (AASHTO)
\item    ``Standard Specifications for Movable Highway Bridges,'' 1988 Edition (AASHTO)
\item    ``Traffic Engineering Metric Conversion Folders---Addendum to the Guide to Metric Conversion,'' 1993 Edition (AASHTO)
\item    ``2009 AREMA Communications \& Signals Manual,'' (American Railway Engineering \& Maintenance-of-Way Association---AREMA)
\item    ``Changeable Message Sign Operation and Messaging Handbook (FHWA-OP-03-070),'' 2004 Edition (Federal Highway Administration---FHWA)
\item    ``Designing Sidewalks and Trails for Access---Part 2---Best Practices Design Guide (FHWA-EP-01-027),'' 2001 Edition (FHWA)
\item    ``Federal-Aid Highway Program Guidance on High Occupancy Vehicle (HOV) Lanes,'' 2001 (FHWA)
\item    ``Maintaining Traffic Sign Retroreflectivity,'' 2007 Edition (FHWA)
\item    ``Railroad-Highway Grade Crossing Handbook---Revised Second Edition (FHWA-SA-07-010),'' 2007 Edition (FHWA)
\item    ``Ramp Management and Control Handbook (FHWA-HOP-06-001),'' 2006 Edition (FHWA)
\item    ``Roundabouts---An Informational Guide (FHWA-RD-00-067),'' 2000 Edition (FHWA)
\item    ``Signal Timing Manual (FHWA-HOP-08-024),'' 2008 Edition (FHWA)
\item    ``Signalized Intersections: an Informational Guide (FHWA-HRT-04-091),'' 2004 Edition (FHWA)
\item    ``Travel Better, Travel Longer: A Pocket Guide to Improving Traffic Control and Mobility for Our Older Population (FHWA-OP-03-098),'' 2003 Edition (FHWA)
\item    ``Practice for Roadway Lighting,'' RP-8, 2001 (Illuminating Engineering Society---IES)
\item    ``Safety Guide for the Prevention of Radio Frequency Radiation Hazards in the Use of Commercial Electric Detonators (Blasting Caps),'' Safety Library Publication No. 20, July 2001 Edition (Institute of Makers of Explosives)
\item    ``American National Standard for High-Visibility Public Safety Vests,'' (ANSI/ISEA 207-2006), 2006 Edition (International Safety Equipment Association---ISEA)
\item    ``American National Standard for High-Visibility Safety Apparel and Headwear,'' (ANSI/ISEA 107-2004), 2004 Edition (ISEA)
\item    ``Manual of Traffic Signal Design,'' 1998 Edition (Institute of Transportation Engineers---ITE)
\item    ``Manual of Transportation Engineering Studies,'' 1994 Edition (ITE)
\item    ``Pedestrian Traffic Control Signal Indications,'' Part 1---1985 Edition; Part 2 (LED Pedestrian Traffic Signal Modules)---2004 Edition (ITE)
\item    ``Preemption of Traffic Signals Near Railroad Crossings,'' 2006 Edition (ITE)
\item    ``Purchase Specification for Flashing and Steady Burn Warning Lights,'' 1981 Edition (ITE)
\item    ``Traffic Control Devices Handbook,'' 2001 Edition (ITE)
\item    ``Traffic Detector Handbook,'' 1991 Edition (ITE)
\item    ``Traffic Engineering Handbook,'' 2009 Edition (ITE)
\item    ``Traffic Signal Lamps,'' 1980 Edition (ITE)
\item    ``Vehicle Traffic Control Signal Heads,'' Part 1---1985 Edition; Part 2 (LED Circular Signal Supplement)---2005 Edition; Part 3 (LED Vehicular Arrow Traffic Signal Supplement)---2004 Edition (ITE)
\item    ``Uniform Vehicle Code (UVC) and Model Traffic Ordinance,'' 2000 Edition (National Committee on Uniform Traffic Laws and Ordinances---NCUTLO)
\item    ``NEMA Standards Publication TS 4-2005 Hardware Standards for Dynamic Message Signs (DMS) With NTCIP Requirements,'' 2005 Edition (National Electrical Manufacturers Association---NEMA)
\item    ``Occupational Safety and Health Administration Regulations (Standards -- 29 CFR), General Safety and Health Provisions -- 1926.20,'' amended June 30, 1993 (Occupational Safety and Health Administration---OSHA)
\item    ``Accessible Pedestrian Signals---A Guide to Best Practices (NCHRP Web-Only Document 117A),'' 2008 Edition (Transportation Research Board---TRB)
\item    ``Guidelines for Accessible Pedestrian Signals (NCHRP Web-Only Document 117B),'' 2008 Edition (TRB)
\item    ``Highway Capacity Manual,'' 2000 Edition (TRB)
\item    ``Recommended Procedures for the Safety Performance Evaluation of Highway Features,'' (NCHRP Report 350), 1993 Edition (TRB)
\item    ``The Americans with Disabilities Act Accessibility Guidelines for Buildings and Facilities (ADAAG),'' July 1998 Edition (The U.S. Access Board)
\end{itemize}}

\section{Color Code}

\support{
The following color code establishes general meanings for 11 colors of a total of 13 colors that have been identified as being appropriate for use in conveying traffic control information. Tolerance limits for each color are contained in the Royal Centralia Comprendium of Sign Fabrication Standards and are available at the Federal Highway Administration's MUTCD website at \url{http://mutcd.fhwa.dot.gov} or by writing to the FHWA, Office of Safety Research and Development (HRD-T-301), 6300 Georgetown Pike, McLean, VA 22101.}

\support{
The two colors for which general meanings have not yet been assigned are being reserved for future applications that will be determined only by FHWA after consultation with the States, the engineering community, and the general public. The meanings described in this Section are of a general nature. More specific assignments of colors are given in the individual Parts of this Manual relating to each class of devices.}

\standard{
The general meaning of the 13 colors shall be as follows:

\begin{enumerate}
\item    Black---regulation
\item    Blue---road user services guidance, tourist information, and evacuation route
\item    Brown---recreational and cultural interest area guidance
\item    Coral---unassigned
\item    Fluorescent Pink---incident management
\item    Fluorescent Yellow-Green---pedestrian warning, bicycle warning, playground warning, school bus and school warning
\item    Green---indicated movements permitted, direction guidance
\item    Light Blue---unassigned
\item    Orange---temporary traffic control
\item    Purple---lanes restricted to use only by vehicles with registered electronic toll collection (ETC) accounts
\item    Red---stop or prohibition
\item    White---regulation
\item    Yellow---warning
\end{enumerate}}

\section{Definitions of Headings, Words, and Phrases in this Manual}
\label{sec:2009.1A.13}
\status{Fill in remainder of this section.}

\standard{
\label{p:2009.1A.13.01}
When used in this Manual, the text headings of Standard, Guidance, Option, and Support shall be defined as follows:
\begin{enumerate}
\item Standard---a statement of required, mandatory, or specifically prohibitive practice regarding a traffic control device. All Standard statements are labeled, and the text appears in bold type. The verb ``shall'' is typically used. The verbs ``should'' and ``may'' are not used in Standard statements. Standard statements are sometimes modified by Options.
\item Guidance---a statement of recommended, but not mandatory, practice in typical situations, with deviations allowed if engineering judgment or engineering study indicates the deviation to be appropriate. All Guidance statements are labeled, and the text appears in unbold type. The verb ``should'' is typically used. The verbs ``shall'' and ``may'' are not used in Guidance statements. Guidance statements are sometimes modified by Options.
\item Option---a statement of practice that is a permissive condition and carries no requirement or recommendation. Option statements sometime contain allowable modifications to a Standard or Guidance statement. All Option statements are labeled, and the text appears in unbold type. The verb ``may'' is typically used. The verbs ``shall'' and ``should'' are not used in Option statements.
\item Support---an informational statement that does not convey any degree of mandate, recommendation, authorization, prohibition, or enforceable condition. Support statements are labeled, and the text appears in unbold type. The verbs ``shall,'' ``should,'' and ``may'' are not used in Support statements.
\end{enumerate}}


\section{Meanings of Acronyms and Abbreviations in this Manual}

section

\section{Abbreviations Used on Traffic Control Devices}

section

\ifstandalone{\end{document}}